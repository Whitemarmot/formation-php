% Formation Niveau 1 - Soudeuse à Points : Les Fondamentaux
% Spot Welding Pro - Kangy Ham
% Version 3.0 - Comprendre, choisir et réussir ses premières soudures

% Préambule commun pour toutes les formations
% Spot Welding Pro - Formations PDF Premium

\documentclass[11pt,a4paper,oneside]{book}

% Encodage et langue
\usepackage[utf8]{inputenc}
\usepackage[T1]{fontenc}
\usepackage[french]{babel}

% Géométrie de page
\usepackage[
    top=2.5cm,
    bottom=2.5cm,
    left=2.5cm,
    right=2.5cm,
    headheight=14pt
]{geometry}

% Polices (utilise les polices TeX Live par défaut)
\usepackage{fontspec}
\setmainfont{Latin Modern Roman}
\setsansfont{Latin Modern Sans}
\setmonofont{Latin Modern Mono}

% Couleurs
\usepackage{xcolor}
\definecolor{primary}{HTML}{E94560}
\definecolor{secondary}{HTML}{F39C12}
\definecolor{darkbg}{HTML}{0F0F1A}
\definecolor{darkcard}{HTML}{16213E}
\definecolor{textcolor}{HTML}{EAEAEA}
\definecolor{mutedtext}{HTML}{9CA3AF}
\definecolor{success}{HTML}{22C55E}
\definecolor{warning}{HTML}{F59E0B}
\definecolor{danger}{HTML}{EF4444}

% Graphiques et images
\usepackage{graphicx}
\usepackage{float}
\usepackage{wrapfig}
\usepackage{caption}
\usepackage{subcaption}

% Tableaux
\usepackage{booktabs}
\usepackage{longtable}
\usepackage{multirow}
\usepackage{makecell}
\usepackage{colortbl}
\usepackage{array}

% Listes
\usepackage{enumitem}
\setlist[itemize]{leftmargin=*,itemsep=0.5em}
\setlist[enumerate]{leftmargin=*,itemsep=0.5em}

% Mathématiques
\usepackage{amsmath}
\usepackage{amssymb}
\usepackage{siunitx}
\sisetup{
    locale=FR,
    output-decimal-marker={,},
    group-separator={\,}
}

% Code et algorithmes
\usepackage{listings}
\lstset{
    basicstyle=\ttfamily\small,
    keywordstyle=\color{primary}\bfseries,
    commentstyle=\color{mutedtext}\itshape,
    stringstyle=\color{success},
    numbers=left,
    numberstyle=\tiny\color{mutedtext},
    numbersep=10pt,
    frame=single,
    frameround=tttt,
    backgroundcolor=\color{darkcard},
    rulecolor=\color{mutedtext},
    breaklines=true,
    showstringspaces=false
}

% Boîtes colorées
\usepackage{tcolorbox}
\tcbuselibrary{skins,breakable}

% Boîte d'information
\newtcolorbox{infobox}[1][]{
    enhanced,
    colback=darkcard,
    colframe=primary,
    coltitle=white,
    fonttitle=\bfseries,
    left=10pt,
    right=10pt,
    top=10pt,
    bottom=10pt,
    arc=4pt,
    boxrule=1pt,
    title=#1
}

% Boîte d'avertissement
\newtcolorbox{warningbox}[1][Attention]{
    enhanced,
    colback=warning!10!darkbg,
    colframe=warning,
    coltitle=white,
    fonttitle=\bfseries,
    left=10pt,
    right=10pt,
    top=10pt,
    bottom=10pt,
    arc=4pt,
    boxrule=1pt,
    title=#1
}

% Boîte de danger
\newtcolorbox{dangerbox}[1][Danger]{
    enhanced,
    colback=danger!10!darkbg,
    colframe=danger,
    coltitle=white,
    fonttitle=\bfseries,
    left=10pt,
    right=10pt,
    top=10pt,
    bottom=10pt,
    arc=4pt,
    boxrule=1pt,
    title=#1
}

% Boîte de conseil
\newtcolorbox{tipbox}[1][Conseil]{
    enhanced,
    colback=success!10!darkbg,
    colframe=success,
    coltitle=white,
    fonttitle=\bfseries,
    left=10pt,
    right=10pt,
    top=10pt,
    bottom=10pt,
    arc=4pt,
    boxrule=1pt,
    title=#1
}

% En-têtes et pieds de page
\usepackage{fancyhdr}
\pagestyle{fancy}
\fancyhf{}
\fancyhead[L]{\small\textcolor{mutedtext}{\leftmark}}
\fancyhead[R]{\small\textcolor{mutedtext}{Spot Welding Pro}}
\fancyfoot[C]{\small\textcolor{mutedtext}{\thepage}}
\renewcommand{\headrulewidth}{0.5pt}
\renewcommand{\headrule}{\hbox to\headwidth{\color{primary}\leaders\hrule height \headrulewidth\hfill}}
\renewcommand{\footrulewidth}{0pt}

% Titres de chapitres
\usepackage{titlesec}
\titleformat{\chapter}[display]
    {\normalfont\huge\bfseries\color{primary}}
    {\chaptertitlename\ \thechapter}
    {20pt}
    {\Huge}
\titleformat{\section}
    {\normalfont\Large\bfseries\color{primary}}
    {\thesection}
    {1em}
    {}
\titleformat{\subsection}
    {\normalfont\large\bfseries}
    {\thesubsection}
    {1em}
    {}
\titleformat{\subsubsection}
    {\normalfont\normalsize\bfseries}
    {\thesubsubsection}
    {1em}
    {}

% Espacement
\usepackage{setspace}
\onehalfspacing

% Table des matières
\usepackage{tocloft}
\renewcommand{\cftchapfont}{\bfseries\color{primary}}
\renewcommand{\cftsecfont}{\color{textcolor}}
\renewcommand{\cftsubsecfont}{\color{mutedtext}}
\renewcommand{\cftchapleader}{\cftdotfill{\cftdotsep}}

% Liens hypertexte
\usepackage[
    colorlinks=true,
    linkcolor=primary,
    urlcolor=secondary,
    citecolor=success,
    bookmarks=true,
    bookmarksnumbered=true
]{hyperref}

% Références croisées améliorées
\usepackage{cleveref}

% Notes de bas de page
\usepackage{footnote}

% Bibliographie
\usepackage[style=numeric,sorting=none]{biblatex}

% Glossaire
\usepackage[acronym,toc]{glossaries}
\makeglossaries

% Index
\usepackage{makeidx}
\makeindex

% Diagrammes TikZ
\usepackage{tikz}
\usetikzlibrary{
    shapes,
    arrows,
    positioning,
    calc,
    decorations.pathreplacing,
    patterns
}

% Circuits électriques
\usepackage{circuitikz}

% Graphiques de données
\usepackage{pgfplots}
\pgfplotsset{compat=1.18}

% Commandes personnalisées
\newcommand{\formation}[1]{\textbf{\textcolor{primary}{#1}}}
\newcommand{\parametre}[1]{\texttt{#1}}
\newcommand{\valeur}[2]{\SI{#1}{#2}}
\newcommand{\marque}[1]{\textit{#1}}
\newcommand{\attention}[1]{\textcolor{warning}{\textbf{#1}}}
\newcommand{\danger}[1]{\textcolor{danger}{\textbf{#1}}}

% Unités personnalisées
\DeclareSIUnit{\ampere}{A}
\DeclareSIUnit{\kiloampere}{kA}
\DeclareSIUnit{\milliseconde}{ms}
\DeclareSIUnit{\newton}{N}
\DeclareSIUnit{\kilonewton}{kN}
\DeclareSIUnit{\ohm}{\Omega}
\DeclareSIUnit{\milliohm}{m\Omega}

% Commande pour la loi de Joule
\newcommand{\joule}{Q = R \cdot I^2 \cdot t}

% Environnement pour les exercices
\newcounter{exercice}[chapter]
\newenvironment{exercice}[1][]{%
    \refstepcounter{exercice}%
    \begin{tcolorbox}[
        enhanced,
        colback=darkcard,
        colframe=secondary,
        coltitle=white,
        fonttitle=\bfseries,
        title={Exercice \theexercice\ifx&#1&\else: #1\fi},
        breakable
    ]
}{%
    \end{tcolorbox}
}

% Environnement pour les études de cas
\newenvironment{casestudy}[1][]{%
    \begin{tcolorbox}[
        enhanced,
        colback=darkcard,
        colframe=primary,
        coltitle=white,
        fonttitle=\bfseries,
        title={Étude de cas\ifx&#1&\else: #1\fi},
        breakable
    ]
}{%
    \end{tcolorbox}
}


% Métadonnées du document
\title{Soudeuse à Points\\Les Fondamentaux}
\author{Kangy Ham}
\date{Version 3.0 - 2025}

\begin{document}

% ============================================
% PAGE DE TITRE
% ============================================
\begin{titlepage}
    \centering
    \vspace*{1cm}

    % Logo / Marque
    {\fontsize{28}{34}\selectfont\textcolor{primary}{\textbf{SPOT WELDING PRO}}}

    \vspace{0.3cm}

    {\large\textcolor{textmuted}{Formations Professionnelles en Soudage}}

    \vspace{2cm}

    % Image de couverture
    \includegraphics[width=0.7\textwidth]{soudeuse-portable.jpg}

    \vspace{1.5cm}

    % Titre principal
    {\fontsize{36}{44}\selectfont\textbf{Soudeuse à Points}}

    \vspace{0.3cm}

    {\Huge\textcolor{primary}{Les Fondamentaux}}

    \vspace{0.8cm}

    {\Large\textit{Comprendre, choisir et réussir ses premières soudures}}

    \vspace{2cm}

    % Auteur
    {\large\textbf{Par Kangy Ham}}

    \vspace{0.2cm}

    {\normalsize Ingénieur Procédés $\cdot$ Expert Batteries Lithium}

    {\small 15 ans d'expérience industrielle}

    \vfill

    % Badge niveau
    \begin{tcolorbox}[
        enhanced,
        colback=primary!10!white,
        colframe=primary,
        width=8cm,
        arc=8pt,
        boxrule=2pt,
        halign=center
    ]
        {\large\textbf{Niveau 1}} $\cdot$ Débutant $\cdot$ \textasciitilde150 pages
    \end{tcolorbox}

    \vspace{0.8cm}

    {\footnotesize © 2025 Spot Welding Pro. Tous droits réservés.}

\end{titlepage}

% ============================================
% PAGE DE COPYRIGHT
% ============================================
\thispagestyle{empty}
\vspace*{\fill}
\begin{center}
    \textbf{\Large Soudeuse à Points --- Les Fondamentaux}

    \vspace{1.5cm}

    © 2025 Spot Welding Pro

    Tous droits réservés.

    \vspace{1.5cm}

    \begin{minipage}{0.8\textwidth}
        Ce document est protégé par le droit d'auteur. Toute reproduction, même partielle,
        est interdite sans autorisation écrite préalable de l'auteur.

        \vspace{1cm}

        \textbf{Avertissement}

        Les informations contenues dans ce document sont fournies à titre éducatif.
        L'auteur décline toute responsabilité en cas d'accident ou de dommage
        résultant de l'application des techniques décrites.

        \vspace{0.5cm}

        \textcolor{danger}{\textbf{Le soudage par points implique des risques électriques et thermiques.
        Respectez toujours les consignes de sécurité applicables.}}
    \end{minipage}

    \vspace{1.5cm}

    \textcolor{textmuted}{
        Normes de référence : ISO 4063:2023, AWS A3.0\\
        Classification du procédé : RSW (21)
    }
\end{center}
\vspace*{\fill}
\newpage

% ============================================
% TABLE DES MATIÈRES
% ============================================
\tableofcontents
\newpage

% ============================================
% LE MOT DU FORMATEUR
% ============================================
\chapter*{Le Mot du Formateur}
\addcontentsline{toc}{chapter}{Le Mot du Formateur}

\begin{wrapfigure}{r}{0.35\textwidth}
    \centering
    \includegraphics[width=0.33\textwidth]{miller-spot-welder.jpg}
    \caption*{\small Une soudeuse professionnelle Miller 12~000A}
\end{wrapfigure}

Cher lecteur,

Si vous tenez ce document entre vos mains, c'est que vous avez fait le choix d'apprendre une technique qui m'a accompagné pendant plus de quinze ans de ma vie professionnelle. Et croyez-moi, c'est un excellent choix.

\textbf{Mon parcours} a commencé en 2008, dans un petit atelier de Toulouse où je réparais des batteries de vélos électriques. À l'époque, je ne comprenais rien à ce que je faisais. Je suivais bêtement les réglages trouvés sur Internet, et mes soudures avaient une qualité... disons variable. Un jour, une de mes batteries a pris feu chez un client. Heureusement, personne n'a été blessé, mais cet événement m'a profondément marqué.

J'ai alors décidé de tout reprendre à zéro. J'ai étudié la métallurgie, la thermodynamique, lu tous les articles scientifiques que je pouvais trouver. J'ai passé des certifications, travaillé pour PSA, Renault, puis Tesla en Californie pendant 3 ans. J'ai analysé des dizaines de milliers de soudures au microscope, testé des centaines de paramètres, et surtout, appris à \textbf{comprendre} avant d'agir.

\subsection*{Ce que cette formation vous apportera}

Ce n'est pas un simple tutoriel. C'est la synthèse de 15 ans d'expérience, de milliers d'heures de pratique, et de toutes les erreurs que j'ai commises pour que vous n'ayez pas à les commettre.

\begin{itemize}
    \item \textbf{La théorie} : parce que comprendre pourquoi une soudure tient (ou pas) est la clé de tout
    \item \textbf{La pratique} : avec des exercices concrets et des paramètres réels
    \item \textbf{Le diagnostic} : savoir identifier et corriger les problèmes rapidement
    \item \textbf{La sécurité} : parce qu'une soudure ratée peut avoir des conséquences graves
\end{itemize}

\subsection*{Ma philosophie}

\begin{tipbox}[Mon conseil le plus important]
Ne jamais souder une pièce sans comprendre ce qui se passe à l'intérieur. Chaque paramètre a une raison d'être. Chaque réglage influence le résultat. Le hasard n'a pas sa place dans une soudure de qualité.
\end{tipbox}

Prenez le temps de lire chaque chapitre. Faites les exercices. Posez-vous des questions. Et surtout, n'hésitez jamais à expérimenter sur des échantillons avant de souder vos pièces définitives.

Bonne lecture et bonne soudure !

\vspace{0.5cm}
\textit{Kangy Ham}

\textit{Janvier 2025}

\newpage

% ============================================
% CHAPITRE 1 : HISTOIRE DU SOUDAGE PAR POINTS
% ============================================
\chapter{Histoire du Soudage par Points}

\begin{objectives}
    \item Découvrir les origines du soudage par résistance
    \item Connaître les pionniers de cette technique
    \item Comprendre l'évolution industrielle du procédé
    \item Situer le soudage par points dans l'histoire de l'industrie moderne
\end{objectives}

Avant de plonger dans les aspects techniques, prenons un moment pour découvrir d'où vient cette technique que vous vous apprêtez à maîtriser. L'histoire du soudage par points est intimement liée à celle de l'électricité industrielle et de l'automobile.

\section{La découverte accidentelle d'Elihu Thomson (1877)}

L'histoire du soudage par résistance commence par un accident. En 1877, à Philadelphie, un jeune ingénieur de 24 ans nommé \textbf{Elihu Thomson} préparait une expérience sur les courants électriques quand deux fils de cuivre se touchèrent accidentellement. Au lieu de simplement créer une étincelle, les fils \textbf{fusionnèrent ensemble}.

\begin{infobox}[Qui était Elihu Thomson ?]
    \textbf{Elihu Thomson} (1853-1937) était un ingénieur et inventeur anglo-américain né à Manchester, en Angleterre. Sa famille émigra aux États-Unis en 1858. Thomson fut un inventeur prolifique avec plus de \textbf{700 brevets} à son actif, ce qui le place juste derrière Thomas Edison en termes de productivité inventive.

    Il cofonda la \textit{Thomson-Houston Electric Company}, qui fusionna avec \textit{Edison General Electric} en 1892 pour former \textbf{General Electric}, l'une des plus grandes entreprises industrielles au monde.
\end{infobox}

Thomson nota cet événement dans son carnet avec la mention ``meilleure chance la prochaine fois'', mais l'idée ne le quitta pas. Et si on pouvait souder des métaux à volonté en utilisant ce phénomène ?

\subsection{Le développement du procédé (1885-1900)}

En 1885, Thomson se consacra sérieusement au développement du soudage par résistance électrique. Son approche était élégamment simple :

\begin{quote}
    ``Tout ce qu'il fallait, c'était un transformateur avec un primaire connecté au circuit d'éclairage et un secondaire de quelques spires de câble de cuivre massif. Les extrémités de ce câble étaient munies de pinces solides qui saisissaient les pièces de métal à souder et les forçaient à se serrer fermement ensemble.''
    --- Elihu Thomson, 1886
\end{quote}

\begin{figure}[H]
    \centering
    \includegraphics[width=0.85\textwidth]{willow-run-bomber-1.jpg}
    \caption{Soudage par points dans l'industrie aéronautique américaine (1942) --- Les techniques développées par Thomson ont permis la production de masse d'avions pendant la Seconde Guerre mondiale.}
\end{figure}

Thomson breveta plusieurs variantes du procédé entre 1885 et 1900 :

\begin{itemize}
    \item \textbf{Soudage par points} (\textit{spot welding}) : deux électrodes pressent les pièces en un point
    \item \textbf{Soudage à la molette} (\textit{seam welding}) : électrodes rotatives pour soudures continues
    \item \textbf{Soudage par bossages} (\textit{projection welding}) : soudure sur des reliefs préformés
    \item \textbf{Soudage bout à bout} (\textit{flash butt welding}) : assemblage de barres ou fils
\end{itemize}

Le premier brevet majeur, U.S. Patent 451,345 ``Method Of Electric Welding'', fut accordé en 1891.

\section{La révolution automobile (1908-1930)}

Le soudage par points aurait pu rester une curiosité de laboratoire si Henry Ford n'avait pas eu une idée révolutionnaire : la \textbf{production à la chaîne}.

\subsection{Le Ford Model T et la naissance de l'assemblage moderne}

En 1908, Ford lança le Model T. L'objectif était ambitieux : produire une voiture que chaque Américain pourrait s'offrir. Pour y parvenir, il fallait assembler des véhicules plus vite et à moindre coût que jamais.

Le soudage par points s'imposa rapidement comme la solution idéale :

\begin{itemize}
    \item \textbf{Rapidité} : une soudure en moins d'une seconde
    \item \textbf{Reproductibilité} : même qualité soudure après soudure
    \item \textbf{Absence de consommable} : pas de baguette, pas de gaz
    \item \textbf{Automatisation facile} : le procédé se prête bien aux machines
\end{itemize}

\begin{defbox}[Production à la chaîne]
    La production à la chaîne (\textit{assembly line}) est un mode de production où le produit se déplace d'un poste de travail à l'autre. Chaque ouvrier effectue une tâche spécifique et répétitive. Cette méthode, perfectionnée par Ford, permit de réduire le temps d'assemblage d'un Model T de 12 heures à 93 minutes.
\end{defbox}

Entre 1908 et 1927, Ford produisit plus de \textbf{15 millions} de Model T, faisant du soudage par points l'un des procédés industriels les plus utilisés au monde.

\subsection{L'entre-deux-guerres : standardisation et amélioration}

Les années 1920-1930 virent une standardisation progressive du procédé :

\begin{itemize}
    \item Développement des premiers \textbf{commutateurs à thyratrons} pour un contrôle précis du courant
    \item Création des premières \textbf{normes industrielles} pour le soudage par résistance
    \item Amélioration des matériaux d'\textbf{électrodes} (passage du cuivre pur aux alliages Cu-Cr)
    \item Introduction des premiers systèmes de \textbf{refroidissement} par eau
\end{itemize}

\section{La Seconde Guerre mondiale : l'âge industriel}

La guerre de 1939-1945 accéléra considérablement le développement du soudage par points. La demande massive en avions, chars et navires nécessitait des techniques d'assemblage rapides et fiables.

\subsection{L'usine Willow Run}

L'exemple le plus spectaculaire est l'usine \textbf{Willow Run} de Ford, près de Detroit. Construite en 1941, cette usine gigantesque (plus de 300~000~m²) produisait des bombardiers B-24 Liberator.

\begin{itemize}
    \item Production maximale : \textbf{un avion toutes les 63 minutes}
    \item Plus de 42~000 ouvriers, dont beaucoup de femmes (les ``Rosie the Riveters'')
    \item Utilisation massive du soudage par points pour l'assemblage des fuselages
\end{itemize}

\begin{warningbox}[Les ``Rosie the Riveters'']
    Pendant la guerre, des millions de femmes américaines prirent des emplois industriels traditionnellement masculins, notamment dans le soudage. ``Rosie the Riveter'' devint un symbole culturel de l'effort de guerre et de l'émancipation féminine. Beaucoup de ces femmes continuèrent à travailler dans l'industrie après la guerre.
\end{warningbox}

\section{L'ère moderne : automatisation et robotique (1960-présent)}

\subsection{L'arrivée des robots industriels}

En 1961, le premier robot industriel, l'\textbf{Unimate}, fut installé dans une usine General Motors au New Jersey. Sa mission ? Le soudage par points et l'extraction de pièces moulées.

\begin{figure}[H]
    \centering
    \includegraphics[width=0.75\textwidth]{robot-soudage.jpg}
    \caption{Robot de soudage par points moderne --- Les robots actuels peuvent réaliser plus de 5~000 points de soudure par carrosserie avec une précision au dixième de millimètre.}
\end{figure}

L'inventeur de l'Unimate, \textbf{George Devol}, avait déposé son brevet en 1954, mais il fallut attendre les années 1980 pour que la robotisation du soudage devienne généralisée.

\subsection{Chiffres clés de l'industrie automobile moderne}

Aujourd'hui, le soudage par points reste le procédé d'assemblage dominant dans l'industrie automobile :

\begin{center}
\begin{tabular}{L{6cm}C{4cm}}
    \toprule
    \textbf{Caractéristique} & \textbf{Valeur typique} \\
    \midrule
    Points de soudure par véhicule & 3~000 à 5~000 \\
    Robots par ligne d'assemblage & 400 à 800 \\
    Temps de cycle par point & 0.2 à 0.5 seconde \\
    Précision de positionnement & ± 0.1 mm \\
    Cadence de production & 60 véhicules/heure \\
    \bottomrule
\end{tabular}
\end{center}

\subsection{L'ère des batteries lithium}

Depuis 2010, le soudage par points connaît une nouvelle révolution avec l'essor des \textbf{véhicules électriques}. Les packs de batteries lithium-ion nécessitent des milliers de soudures pour connecter les cellules individuelles.

\begin{figure}[H]
    \centering
    \includegraphics[width=0.8\textwidth]{usine-bmw.jpg}
    \caption{Ligne de production robotisée chez BMW Leipzig --- L'assemblage des batteries électriques nécessite des milliers de soudures par points avec une précision et une reproductibilité parfaites.}
\end{figure}

Cette application a également démocratisé le soudage par points auprès des \textbf{makers} et bricoleurs, qui peuvent désormais assembler leurs propres packs de batteries pour vélos électriques, powerwalls ou autres projets.

\begin{keypoints}
    \item Le soudage par résistance fut découvert accidentellement par Elihu Thomson en 1877
    \item Thomson breveta le procédé entre 1885 et 1900, développant plusieurs variantes
    \item Ford popularisa le soudage par points avec la production de masse du Model T
    \item La Seconde Guerre mondiale accéléra le développement industriel du procédé
    \item Le premier robot de soudage (Unimate) fut installé en 1961 chez General Motors
    \item Aujourd'hui, une voiture moderne contient 3~000 à 5~000 points de soudure
    \item L'essor des batteries lithium a démocratisé le soudage par points
\end{keypoints}

\newpage

% ============================================
% CHAPITRE 2 : PHYSIQUE DU SOUDAGE PAR RÉSISTANCE
% ============================================
\chapter{Physique du Soudage par Résistance}

\begin{objectives}
    \item Comprendre le principe de l'effet Joule
    \item Identifier les zones de résistance dans une soudure
    \item Connaître les paramètres qui influencent la qualité
    \item Distinguer les différents matériaux soudables
\end{objectives}

Maintenant que vous connaissez l'histoire du procédé, plongeons dans les fondamentaux physiques. Cette compréhension est \textbf{essentielle} : elle vous permettra d'ajuster vos paramètres de manière rationnelle plutôt que par tâtonnement.

\section{L'effet Joule : le c\oe{}ur du procédé}

Le soudage par points repose sur un phénomène physique découvert par James Prescott Joule en 1841 : \textbf{l'effet Joule}.

\begin{defbox}[Loi de Joule]
    Lorsqu'un courant électrique traverse un conducteur, une partie de l'énergie électrique est convertie en chaleur. La quantité de chaleur produite est donnée par :
    \begin{formulabox}
        \begin{equation}
            Q = R \cdot I^2 \cdot t
        \end{equation}
    \end{formulabox}
    Où :
    \begin{itemize}
        \item $Q$ = chaleur produite (en Joules)
        \item $R$ = résistance électrique (en Ohms, $\Omega$)
        \item $I$ = intensité du courant (en Ampères, A)
        \item $t$ = durée du passage du courant (en secondes, s)
    \end{itemize}
\end{defbox}

\subsection{Analyse de la formule}

Cette formule simple contient toute la clé du soudage par résistance. Analysons chaque terme :

\subsubsection{L'intensité au carré ($I^2$)}

Le courant apparaît au carré. Cela signifie que si vous doublez le courant, vous quadruplez la chaleur produite. C'est pourquoi :

\begin{itemize}
    \item Les soudeuses industrielles utilisent des courants de plusieurs milliers d'ampères
    \item Un réglage de courant trop élevé peut instantanément détruire une soudure
    \item La précision du réglage du courant est cruciale
\end{itemize}

\begin{warningbox}[Attention au courant !]
    Un courant de 8~000~A au lieu de 6~000~A ne représente qu'une augmentation de 33\%, mais la chaleur produite augmente de 78\% ! C'est pourquoi les ajustements de courant doivent toujours être progressifs.
\end{warningbox}

\subsubsection{La résistance ($R$)}

La résistance détermine \textbf{où} la chaleur est produite. Et c'est là que les choses deviennent intéressantes...

\subsubsection{Le temps ($t$)}

Le temps est le paramètre le plus facile à contrôler. La chaleur est directement proportionnelle à la durée : doubler le temps double la chaleur.

\section{Les sept zones de résistance}

Quand vous pressez deux électrodes sur deux tôles à souder, le courant traverse \textbf{sept zones de résistance} distinctes :

\begin{figure}[H]
    \centering
    \begin{tikzpicture}[scale=1.0]
        % Électrode supérieure
        \fill[electrode] (-1,3.5) rectangle (1,5);
        \draw[thick] (-1,3.5) -- (-1,5) -- (1,5) -- (1,3.5);
        \node at (0,4.5) {\textbf{Électrode}};
        \node at (0,4) {\textbf{supérieure}};

        % Tôle supérieure
        \fill[metal] (-3,2) rectangle (3,3.5);
        \draw[thick] (-3,2) -- (-3,3.5) -- (3,3.5) -- (3,2);
        \node at (-2,2.75) {Tôle 1};

        % Tôle inférieure
        \fill[metal] (-3,0.5) rectangle (3,2);
        \draw[thick] (-3,0.5) -- (-3,2) -- (3,2) -- (3,0.5);
        \node at (-2,1.25) {Tôle 2};

        % Électrode inférieure
        \fill[electrode] (-1,-1) rectangle (1,0.5);
        \draw[thick] (-1,-1) -- (-1,0.5) -- (1,0.5) -- (1,-1);
        \node at (0,-0.5) {\textbf{Électrode}};
        \node at (0,0) {\textbf{inférieure}};

        % Zones de résistance avec numéros
        \draw[primary, ultra thick, <->] (1.5,4.25) -- (2.5,4.25);
        \node[right] at (2.5,4.25) {$R_1$ : Électrode sup.};

        \draw[primary, ultra thick, <->] (1.5,3.5) -- (2.5,3.5);
        \node[right] at (2.5,3.5) {\textcolor{warning}{$R_2$ : Contact élec./tôle}};

        \draw[primary, ultra thick, <->] (1.5,2.75) -- (2.5,2.75);
        \node[right] at (2.5,2.75) {$R_3$ : Tôle supérieure};

        \draw[danger, ultra thick, <->] (1.5,2) -- (2.5,2);
        \node[right] at (2.5,2) {\textcolor{danger}{\textbf{$R_4$ : Interface (SOUDURE)}}};

        \draw[primary, ultra thick, <->] (1.5,1.25) -- (2.5,1.25);
        \node[right] at (2.5,1.25) {$R_5$ : Tôle inférieure};

        \draw[primary, ultra thick, <->] (1.5,0.5) -- (2.5,0.5);
        \node[right] at (2.5,0.5) {\textcolor{warning}{$R_6$ : Contact élec./tôle}};

        \draw[primary, ultra thick, <->] (1.5,-0.25) -- (2.5,-0.25);
        \node[right] at (2.5,-0.25) {$R_7$ : Électrode inf.};

        % Point de soudure
        \fill[weld] (0,2) ellipse (0.8 and 0.4);

        % Flèches de courant
        \draw[arrow, ultra thick] (0,5.5) -- (0,5);
        \node[above] at (0,5.5) {\textbf{Courant I}};
        \draw[arrow, ultra thick] (0,-1) -- (0,-1.5);
    \end{tikzpicture}
    \caption{Les sept zones de résistance dans une soudure par points}
\end{figure}

\subsection{La résistance critique : l'interface ($R_4$)}

La zone la plus importante est $R_4$, l'interface entre les deux tôles. C'est là que doit se concentrer la chaleur pour créer le noyau de soudure (\textit{nugget}).

Pourquoi cette zone a-t-elle une résistance élevée ?

\begin{itemize}
    \item \textbf{Contact imparfait} : même pressées, les surfaces ne sont en contact qu'en quelques micro-points
    \item \textbf{Oxydes} : une fine couche d'oxyde recouvre généralement les surfaces métalliques
    \item \textbf{Rugosité} : les irrégularités de surface limitent les zones de contact réel
\end{itemize}

\begin{figure}[H]
    \centering
    \includegraphics[width=0.7\textwidth]{schema-noyau-soudure.png}
    \caption{Coupe schématique d'un point de soudure montrant le noyau fondu (nugget) et la zone affectée thermiquement (ZAT)}
\end{figure}

\subsection{Les résistances de contact électrode/tôle ($R_2$ et $R_6$)}

Ces résistances sont problématiques car elles génèrent de la chaleur \textbf{au mauvais endroit}. Si elles sont trop élevées :

\begin{itemize}
    \item Les électrodes surchauffent et s'usent prématurément
    \item La surface des tôles peut être marquée ou brûlée
    \item L'énergie n'arrive pas efficacement à l'interface
\end{itemize}

Solutions pour minimiser $R_2$ et $R_6$ :

\begin{itemize}
    \item Utiliser des électrodes propres et bien entretenues
    \item Appliquer une force suffisante pour un bon contact
    \item Nettoyer les surfaces des tôles si nécessaire
\end{itemize}

\section{Le diagramme de résistance dynamique}

La résistance de l'assemblage n'est pas constante pendant la soudure. Elle évolue selon un schéma caractéristique :

\begin{figure}[H]
    \centering
    \includegraphics[width=0.85\textwidth]{resistance-dynamique.png}
    \caption{Évolution de la résistance dynamique pendant un cycle de soudure --- Ce graphique permet de suivre la formation du noyau en temps réel.}
\end{figure}

\subsection{Les phases de la résistance dynamique}

\begin{enumerate}
    \item \textbf{Phase initiale} : la résistance est maximale (contacts froids, oxydes)
    \item \textbf{Chute rapide} : les aspérités s'écrasent, les oxydes se percent
    \item \textbf{Minimum} : bon contact établi, métal encore solide
    \item \textbf{Remontée} : le métal chauffe, sa résistivité augmente
    \item \textbf{Plateau ou légère baisse} : formation du noyau liquide
    \item \textbf{Fin de soudure} : arrêt du courant, refroidissement
\end{enumerate}

\begin{infobox}[Diagnostic par résistance dynamique]
    Les soudeuses modernes mesurent la résistance dynamique en temps réel. L'analyse de cette courbe permet de détecter automatiquement les défauts : expulsion, collage, soudure froide... Cette technique est appelée \textbf{contrôle adaptatif}.
\end{infobox}

\section{Les quatre paramètres fondamentaux}

Le soudage par points est contrôlé par quatre paramètres principaux, souvent mémorisés par l'acronyme \textbf{ITFM} :

\begin{center}
\begin{tabular}{C{2cm}L{4cm}L{6cm}}
    \toprule
    \textbf{Paramètre} & \textbf{Description} & \textbf{Effet sur la soudure} \\
    \midrule
    \textbf{I} & Intensité (kA) & Chaleur produite ($\propto I^2$) \\
    \textbf{T} & Temps (ms) & Durée d'apport d'énergie \\
    \textbf{F} & Force (daN) & Qualité du contact, expulsion \\
    \textbf{M} & Matériau & Résistivité, conductivité thermique \\
    \bottomrule
\end{tabular}
\end{center}

\subsection{L'intensité (I)}

C'est le paramètre le plus influent. Les valeurs typiques sont :

\begin{itemize}
    \item \textbf{Nickel 0.15~mm} : 800 à 1~500~A
    \item \textbf{Acier 1~mm} : 6~000 à 10~000~A
    \item \textbf{Acier 2~mm} : 10~000 à 14~000~A
    \item \textbf{Aluminium 1~mm} : 20~000 à 30~000~A
\end{itemize}

\subsection{Le temps (T)}

Exprimé en millisecondes (ms) ou en périodes (1 période = 20~ms à 50~Hz). Valeurs typiques :

\begin{itemize}
    \item \textbf{Batteries lithium} : 5 à 20~ms
    \item \textbf{Tôles fines (< 1~mm)} : 100 à 200~ms
    \item \textbf{Tôles épaisses (> 2~mm)} : 300 à 500~ms
\end{itemize}

\subsection{La force (F)}

La force de serrage des électrodes a plusieurs fonctions :

\begin{itemize}
    \item \textbf{Assurer le contact électrique} entre les pièces
    \item \textbf{Contenir le métal fondu} pour éviter les projections
    \item \textbf{Forger la soudure} pendant le refroidissement
\end{itemize}

Valeurs typiques : 100 à 500~daN pour les tôles d'acier de 1 à 2~mm.

\begin{warningbox}[Force insuffisante = Projections !]
    Si la force est trop faible, le métal fondu n'est pas contenu et s'échappe en projections (expulsion). C'est l'un des défauts les plus courants chez les débutants.
\end{warningbox}

\section{Propriétés des matériaux}

Tous les métaux ne se soudent pas de la même façon. Deux propriétés sont déterminantes :

\subsection{La résistivité électrique}

Plus un métal est résistif, plus il chauffe facilement. Ordre de grandeur :

\begin{center}
\begin{tabular}{L{4cm}C{4cm}}
    \toprule
    \textbf{Matériau} & \textbf{Résistivité ($\mu\Omega \cdot cm$)} \\
    \midrule
    Cuivre & 1.7 \\
    Aluminium & 2.8 \\
    Nickel & 7.0 \\
    Acier doux & 15 - 20 \\
    Acier inoxydable & 70 - 80 \\
    \bottomrule
\end{tabular}
\end{center}

\subsection{La conductivité thermique}

Plus un métal conduit bien la chaleur, plus il est difficile à souder (la chaleur s'échappe rapidement).

\begin{infobox}[Pourquoi l'aluminium est difficile ?]
    L'aluminium combine une faible résistivité (peu de chaleur produite) et une excellente conductivité thermique (la chaleur s'échappe vite). Il faut donc des courants très élevés (20-30~kA) et des temps très courts (quelques ms) pour le souder.
\end{infobox}

\begin{keypoints}
    \item Le soudage par points repose sur l'effet Joule : $Q = R \cdot I^2 \cdot t$
    \item Le courant a l'effet le plus fort (au carré) sur la chaleur produite
    \item La soudure se forme à l'interface entre les tôles ($R_4$)
    \item Les quatre paramètres fondamentaux sont : Intensité, Temps, Force, Matériau
    \item La résistance dynamique évolue pendant la soudure et permet le diagnostic
    \item La résistivité et la conductivité thermique déterminent la soudabilité d'un métal
\end{keypoints}

\newpage

% ============================================
% CHAPITRE 3 : ÉQUIPEMENTS ET COMPOSANTS
% ============================================
\chapter{Équipements et Composants}

\begin{objectives}
    \item Connaître les différents types de soudeuses par points
    \item Identifier les composants d'une soudeuse
    \item Choisir l'équipement adapté à vos besoins
    \item Comprendre les caractéristiques techniques clés
\end{objectives}

Maintenant que vous comprenez la physique du soudage, découvrons les machines qui mettent cette physique en application.

\section{Architecture d'une soudeuse par points}

Toute soudeuse par points, qu'elle soit industrielle ou de bricolage, comprend les mêmes composants fondamentaux :

\begin{figure}[H]
    \centering
    \includegraphics[width=0.75\textwidth]{schema-poste.png}
    \caption{Schéma de principe d'une soudeuse par points}
\end{figure}

\subsection{Le transformateur}

C'est le cœur de la soudeuse. Son rôle est de convertir la tension du secteur (230~V) en une tension très basse (1 à 3~V) tout en multipliant le courant.

\begin{defbox}[Principe du transformateur]
    Un transformateur est constitué de deux bobines (primaire et secondaire) enroulées autour d'un noyau magnétique. Le rapport de transformation est :
    $$\frac{U_1}{U_2} = \frac{N_1}{N_2}$$
    Où $U$ est la tension et $N$ le nombre de spires. La puissance étant conservée ($U_1 \cdot I_1 \approx U_2 \cdot I_2$), réduire la tension augmente proportionnellement le courant.
\end{defbox}

Exemple concret : un transformateur avec un primaire de 230~V/20~A et un secondaire de 2~V peut théoriquement fournir 2~300~A.

\subsection{Le système de commande}

Il gère la durée et le séquençage du courant. On distingue :

\begin{itemize}
    \item \textbf{Commande manuelle} : simple interrupteur ou pédale
    \item \textbf{Temporisation analogique} : potentiomètre réglant le temps
    \item \textbf{Commande numérique} : microcontrôleur (Arduino, etc.) pour un contrôle précis
    \item \textbf{Commande adaptative} : ajustement en temps réel selon la résistance mesurée
\end{itemize}

\subsection{Les électrodes}

Les électrodes transmettent le courant et la pression aux pièces. Leurs caractéristiques sont critiques :

\begin{figure}[H]
    \centering
    \includegraphics[width=0.65\textwidth]{electrodes-types.jpg}
    \caption{Différentes formes d'électrodes pour soudage par points}
\end{figure}

\begin{center}
\begin{tabular}{L{3cm}L{4cm}L{5cm}}
    \toprule
    \textbf{Type} & \textbf{Application} & \textbf{Caractéristiques} \\
    \midrule
    Pointe tronquée & Usage général & Polyvalente, facile à affûter \\
    Dome (bombée) & Tôles fines & Bon contact, marques discrètes \\
    Plate & Grandes surfaces & Courant réparti, moins de marquage \\
    Offset (coudée) & Zones difficiles & Accès aux endroits exigus \\
    \bottomrule
\end{tabular}
\end{center}

\subsubsection{Matériaux des électrodes}

\begin{itemize}
    \item \textbf{Cuivre pur (Cu)} : excellente conductivité, mais s'use vite
    \item \textbf{Cuivre-chrome (CuCr)} : bon compromis conductivité/dureté
    \item \textbf{Cuivre-chrome-zirconium (CuCrZr)} : haute dureté, pour grandes séries
    \item \textbf{Tungstène-cuivre (WCu)} : pour aluminium et matériaux réfractaires
\end{itemize}

\subsection{Le système de pression}

La force de serrage peut être générée par :

\begin{itemize}
    \item \textbf{Pince manuelle} : effort de l'opérateur (soudeuses portables)
    \item \textbf{Ressort calibré} : force constante prédéfinie
    \item \textbf{Vérin pneumatique} : force réglable, répétable (industrie)
    \item \textbf{Servo-électrique} : contrôle précis en boucle fermée
\end{itemize}

\subsection{Le refroidissement}

Pour les soudeuses intensives, un système de refroidissement est indispensable :

\begin{itemize}
    \item \textbf{Air (convection naturelle)} : soudeuses légères, usage intermittent
    \item \textbf{Eau} : circulation dans les électrodes, usage continu
    \item \textbf{Groupe froid} : eau réfrigérée pour les applications exigeantes
\end{itemize}

\section{Types de soudeuses par points}

\subsection{Soudeuses sur pied (stationnaires)}

\begin{figure}[H]
    \centering
    \includegraphics[width=0.55\textwidth]{soudeuse-pedestre.jpg}
    \caption{Soudeuse à points sur pied --- machine professionnelle pour atelier}
\end{figure}

Ces machines sont conçues pour un usage intensif en atelier :

\begin{itemize}
    \item Puissance : 15 à 50~kVA
    \item Courant : 8~000 à 20~000~A
    \item Bras : ouverture de 300 à 600~mm
    \item Commande : pédale ou bimanuelle
    \item Refroidissement : eau obligatoire
\end{itemize}

\textbf{Applications} : carrosserie automobile, chaudronnerie, mobilier métallique.

\subsection{Soudeuses portables (pinces)}

\begin{figure}[H]
    \centering
    \includegraphics[width=0.6\textwidth]{soudeuse-portable.jpg}
    \caption{Soudeuse portable à pince --- idéale pour les travaux de carrosserie sur site}
\end{figure}

Les pinces portables sont utilisées pour les travaux en position :

\begin{itemize}
    \item Puissance : 5 à 15~kVA
    \item Courant : 3~000 à 10~000~A
    \item Poids : 15 à 40~kg
    \item Câbles : alimentation séparée du transformateur
\end{itemize}

\textbf{Applications} : réparation automobile, chantiers navals, maintenance.

\subsection{Soudeuses pour batteries (micro-soudage)}

C'est la catégorie qui vous intéresse probablement le plus si vous êtes maker :

\begin{figure}[H]
    \centering
    \includegraphics[width=0.7\textwidth]{miller-spot-welder.jpg}
    \caption{Soudeuse professionnelle Miller --- modèle haut de gamme pour applications industrielles}
\end{figure}

Caractéristiques :

\begin{itemize}
    \item Puissance : 0.5 à 3~kVA
    \item Courant : 500 à 3~000~A
    \item Temps de soudure : 1 à 50~ms
    \item Alimentation : secteur ou batterie
\end{itemize}

\textbf{Applications} : assemblage de packs batteries, électronique, bijouterie.

\section{Caractéristiques techniques à comparer}

Lors du choix d'une soudeuse, examinez attentivement ces paramètres :

\begin{center}
\begin{tabular}{L{4cm}L{4cm}L{4cm}}
    \toprule
    \textbf{Paramètre} & \textbf{Entrée de gamme} & \textbf{Professionnel} \\
    \midrule
    Courant max & 1~000 - 2~000~A & 5~000 - 20~000~A \\
    Contrôle temps & 1 - 99 ms & 0.1 ms résolution \\
    Force électrodes & Manuelle & Pneumatique réglable \\
    Cycle de travail & 10 - 20\% & 50 - 100\% \\
    Refroidissement & Air & Eau \\
    Prix indicatif & 50 - 300~€ & 2~000 - 50~000~€ \\
    \bottomrule
\end{tabular}
\end{center}

\begin{warningbox}[Le piège du courant affiché]
    Méfiez-vous des soudeuses bon marché qui affichent des courants impressionnants (``10~000~A !''). Ce qui compte, c'est le courant \textbf{réellement délivré aux électrodes}, qui dépend de la qualité du transformateur et des câbles. Une soudeuse de qualité avec 2~000~A effectifs soudera mieux qu'une soudeuse médiocre affichant 5~000~A.
\end{warningbox}

\section{Accessoires essentiels}

\subsection{Pointes et électrodes de rechange}

Les électrodes s'usent. Prévoyez :

\begin{itemize}
    \item Un jeu de pointes de différentes formes
    \item Une lime ou un outil d'affûtage
    \item Des électrodes de remplacement
\end{itemize}

\subsection{Bandes et fils de nickel}

Pour l'assemblage de batteries, vous aurez besoin de :

\begin{itemize}
    \item \textbf{Nickel pur} : 0.1~mm, 0.15~mm, 0.2~mm d'épaisseur
    \item \textbf{Nickel plaqué acier} : moins cher, résistance mécanique supérieure
    \item Largeurs courantes : 5~mm, 8~mm, 10~mm
\end{itemize}

\subsection{Équipements de mesure}

\begin{itemize}
    \item \textbf{Pince ampèremétrique} : pour mesurer le courant réel
    \item \textbf{Dynamomètre} : pour vérifier la force des électrodes
    \item \textbf{Loupe ou microscope} : pour inspecter les soudures
\end{itemize}

\begin{keypoints}
    \item Une soudeuse comprend : transformateur, commande, électrodes, système de pression
    \item Les électrodes en alliage cuivre-chrome offrent le meilleur compromis
    \item Les soudeuses sur pied sont pour l'atelier, les pinces pour le travail en position
    \item Le courant affiché n'est pas toujours le courant réel délivré
    \item Prévoyez des consommables : électrodes, bandes de nickel, outils d'affûtage
\end{keypoints}

\newpage

% ============================================
% CHAPITRE 4 : SÉCURITÉ
% ============================================
\chapter{Sécurité}

\begin{objectives}
    \item Identifier les risques du soudage par points
    \item Connaître les équipements de protection individuelle (EPI)
    \item Mettre en place un poste de travail sécurisé
    \item Réagir correctement en cas d'incident
\end{objectives}

\begin{dangerbox}[Ce chapitre peut vous sauver la vie]
    Le soudage par points manipule des courants de plusieurs milliers d'ampères et génère des températures supérieures à 1~500°C. Chaque année, des accidents graves surviennent par négligence des règles de sécurité. Lisez ce chapitre attentivement.
\end{dangerbox}

\section{Les risques du soudage par points}

\subsection{Risques électriques}

Contrairement à ce qu'on pourrait croire, le risque d'électrocution directe est \textbf{faible} en soudage par points. La tension secondaire (1 à 3~V) est bien en dessous du seuil de danger.

\textbf{Cependant}, les risques électriques existent :

\begin{itemize}
    \item \textbf{Côté primaire} : le branchement secteur (230~V) est dangereux
    \item \textbf{Courts-circuits} : peuvent provoquer des arcs et des incendies
    \item \textbf{Batteries lithium} : tension dangereuse dans les packs en cours d'assemblage
\end{itemize}

\subsection{Risques thermiques}

Les températures impliquées sont extrêmes :

\begin{itemize}
    \item \textbf{Noyau de soudure} : 1~400 à 1~600°C (acier)
    \item \textbf{Projections} : gouttelettes de métal fondu à plus de 1~000°C
    \item \textbf{Électrodes} : peuvent atteindre 200-300°C en usage continu
    \item \textbf{Pièces soudées} : restent chaudes plusieurs secondes
\end{itemize}

\begin{warningbox}[Brûlures retardées]
    Les pièces métalliques peuvent sembler froides alors qu'elles sont encore à 100-150°C. Attendez toujours quelques secondes avant de manipuler une pièce venant d'être soudée.
\end{warningbox}

\subsection{Projections et éclaboussures}

Les expulsions (\textit{splash}) projettent des particules de métal fondu à grande vitesse :

\begin{itemize}
    \item Direction imprévisible (souvent latérale)
    \item Température supérieure à 1~000°C
    \item Peuvent atteindre plusieurs mètres
    \item Risque d'incendie sur matériaux inflammables
\end{itemize}

\subsection{Fumées et vapeurs}

Le soudage génère des fumées contenant :

\begin{itemize}
    \item \textbf{Oxydes métalliques} : fer, zinc, chrome (acier inox)
    \item \textbf{Vapeurs de revêtement} : zinc (galvanisé), peinture, huile
    \item \textbf{Gaz} : ozone (formé par les étincelles)
\end{itemize}

\begin{infobox}[Fièvre des fondeurs]
    L'inhalation de vapeurs de zinc (soudage de tôles galvanisées) peut provoquer la ``fièvre des fondeurs'' : frissons, fièvre, douleurs musculaires apparaissant 4 à 8 heures après l'exposition. Les symptômes disparaissent en 24-48h, mais une ventilation adéquate est indispensable.
\end{infobox}

\subsection{Risques spécifiques aux batteries lithium}

L'assemblage de batteries lithium présente des risques particuliers :

\begin{itemize}
    \item \textbf{Court-circuit} : emballement thermique (\textit{thermal runaway})
    \item \textbf{Soudure traversante} : perforation de la cellule
    \item \textbf{Surchauffe} : dégradation de la cellule, dégazage
    \item \textbf{Incendie} : les batteries lithium brûlent violemment
\end{itemize}

\begin{dangerbox}[Règle absolue pour les batteries lithium]
    Ne JAMAIS souder directement sur le corps d'une cellule lithium-ion. Utilisez uniquement des cellules avec languettes pré-soudées ou des cellules spécifiquement conçues pour le soudage direct (cellules ``weldable'').
\end{dangerbox}

\section{Équipements de protection individuelle (EPI)}

\subsection{Protection des yeux}

\begin{itemize}
    \item \textbf{Lunettes de sécurité} : obligatoires (norme EN 166)
    \item Verres teintés non nécessaires (contrairement au soudage à l'arc)
    \item Protection latérale recommandée contre les projections
\end{itemize}

\subsection{Protection des mains}

\begin{itemize}
    \item \textbf{Gants de travail} : cuir ou matériau anti-chaleur
    \item Éviter les gants trop épais qui gênent la dextérité
    \item Ne pas porter de bagues ou bracelets métalliques
\end{itemize}

\subsection{Protection du corps}

\begin{itemize}
    \item \textbf{Vêtements} : coton ou matériaux ignifugés, manches longues
    \item \textbf{Éviter} : synthétiques (fondent), vêtements amples (accrochage)
    \item \textbf{Chaussures} : fermées, semelles isolantes
    \item \textbf{Tablier} : recommandé pour usage intensif
\end{itemize}

\subsection{Protection respiratoire}

\begin{itemize}
    \item \textbf{Ventilation naturelle} : minimum pour travaux occasionnels
    \item \textbf{Extraction locale} : bras aspirant au plus près de la soudure
    \item \textbf{Masque FFP2/FFP3} : si ventilation insuffisante
\end{itemize}

\section{Organisation du poste de travail}

\subsection{Zone de travail dégagée}

\begin{itemize}
    \item Rayon de 2 mètres sans matériaux inflammables
    \item Sol propre et sec (pas de flaques d'eau ou d'huile)
    \item Éclairage suffisant (500 lux minimum)
    \item Accès dégagé aux issues de secours
\end{itemize}

\subsection{Équipements de sécurité à portée}

\begin{itemize}
    \item \textbf{Extincteur} : CO2 ou poudre (pas d'eau sur feux électriques !)
    \item \textbf{Seau de sable} : pour batteries lithium en feu
    \item \textbf{Couverture anti-feu} : pour étouffer les flammes
    \item \textbf{Trousse de premiers secours} : avec traitement des brûlures
\end{itemize}

\subsection{Spécificités batteries lithium}

Pour l'assemblage de batteries :

\begin{itemize}
    \item Travailler sur surface non conductrice
    \item Isoler les cellules non utilisées
    \item Avoir un récipient de sable ou vermiculite à proximité
    \item Ne jamais laisser un pack en charge sans surveillance
\end{itemize}

\section{Procédures d'urgence}

\subsection{En cas de brûlure}

\begin{enumerate}
    \item Refroidir immédiatement à l'eau froide (15-20 minutes)
    \item Ne pas appliquer de glace directement
    \item Ne pas percer les cloques
    \item Consulter un médecin si la brûlure est étendue ou profonde
\end{enumerate}

\subsection{En cas d'incendie de batterie lithium}

\begin{enumerate}
    \item Évacuer la zone immédiatement
    \item Ne PAS utiliser d'eau (réaction violente avec le lithium)
    \item Utiliser du sable, de la vermiculite ou un extincteur classe D
    \item Appeler les secours
    \item Ventiler la zone (gaz toxiques)
\end{enumerate}

\begin{keypoints}
    \item Le risque principal n'est pas l'électrocution mais les brûlures et projections
    \item Lunettes de sécurité et gants sont obligatoires
    \item Ventiler le poste de travail, surtout pour les tôles galvanisées
    \item Les batteries lithium présentent des risques spécifiques (emballement thermique)
    \item Avoir un extincteur CO2 et du sable à portée de main
    \item Ne jamais souder directement sur le corps d'une cellule lithium
\end{keypoints}

\newpage

% ============================================
% CHAPITRE 5 : PRATIQUE DU SOUDAGE
% ============================================
\chapter{Pratique du Soudage}

\begin{objectives}
    \item Préparer correctement les pièces à souder
    \item Régler les paramètres de soudage
    \item Réaliser vos premières soudures
    \item Contrôler la qualité des soudures
\end{objectives}

Vous avez maintenant les bases théoriques et les connaissances de sécurité. Il est temps de passer à la pratique !

\section{Préparation des pièces}

La qualité d'une soudure commence \textbf{avant} la soudure elle-même.

\subsection{État de surface}

Les surfaces doivent être :

\begin{itemize}
    \item \textbf{Propres} : pas de graisse, huile, poussière
    \item \textbf{Sèches} : pas d'humidité
    \item \textbf{Décapées} : si couche d'oxyde importante
    \item \textbf{Planes} : bon contact sur toute la surface
\end{itemize}

\begin{tipbox}[Nettoyage efficace]
    Pour les tôles légèrement grasses, un simple chiffon imbibé d'alcool isopropylique suffit. Pour les oxydes tenaces, utilisez une brosse métallique ou du papier abrasif fin (grain 400).
\end{tipbox}

\subsection{Positionnement}

\begin{itemize}
    \item Les pièces doivent être \textbf{bien en contact} (pas d'espace)
    \item Utiliser des serre-joints ou un gabarit si nécessaire
    \item Vérifier l'alignement avant de souder
\end{itemize}

\subsection{Cas des batteries}

Pour les cellules lithium :

\begin{itemize}
    \item Vérifier la tension de chaque cellule (équilibrage)
    \item S'assurer que les languettes sont propres
    \item Positionner les bandes de nickel correctement
    \item Ne pas appuyer sur le corps de la cellule
\end{itemize}

\section{Réglage des paramètres}

\subsection{Approche méthodique}

Ne devinez pas les paramètres. Procédez ainsi :

\begin{enumerate}
    \item Identifier le matériau et l'épaisseur des pièces
    \item Consulter les tables de paramètres (voir Annexe)
    \item Commencer par des valeurs \textbf{conservatrices} (courant bas)
    \item Faire des essais sur échantillons
    \item Augmenter progressivement jusqu'au résultat souhaité
\end{enumerate}

\subsection{Table de paramètres de départ}

Pour le soudage de bandes de nickel sur cellules lithium :

\begin{center}
\begin{tabular}{C{2.5cm}C{2.5cm}C{2.5cm}C{2.5cm}}
    \toprule
    \textbf{Épaisseur Ni} & \textbf{Courant} & \textbf{Temps} & \textbf{Force} \\
    \midrule
    0.10 mm & 600 - 800 A & 3 - 5 ms & Légère \\
    0.15 mm & 800 - 1~200 A & 5 - 10 ms & Moyenne \\
    0.20 mm & 1~000 - 1~500 A & 8 - 15 ms & Ferme \\
    0.25 mm & 1~200 - 1~800 A & 10 - 20 ms & Forte \\
    \bottomrule
\end{tabular}
\end{center}

Pour les tôles d'acier :

\begin{center}
\begin{tabular}{C{2.5cm}C{2.5cm}C{2.5cm}C{2.5cm}}
    \toprule
    \textbf{Épaisseur} & \textbf{Courant} & \textbf{Temps} & \textbf{Force} \\
    \midrule
    0.5 mm & 4~000 - 5~000 A & 80 - 100 ms & 150 daN \\
    1.0 mm & 6~000 - 8~000 A & 150 - 200 ms & 250 daN \\
    1.5 mm & 8~000 - 10~000 A & 200 - 300 ms & 350 daN \\
    2.0 mm & 10~000 - 12~000 A & 300 - 400 ms & 450 daN \\
    \bottomrule
\end{tabular}
\end{center}

\begin{warningbox}[Ces valeurs sont indicatives !]
    Les paramètres optimaux dépendent de votre matériel spécifique, de l'état des électrodes, du matériau exact... Faites TOUJOURS des essais sur échantillons avant de souder vos pièces définitives.
\end{warningbox}

\section{Technique de soudage}

\subsection{Séquence de base}

\begin{enumerate}
    \item \textbf{Positionner} les électrodes perpendiculairement à la pièce
    \item \textbf{Appliquer la force} de serrage
    \item \textbf{Déclencher} le cycle de soudage
    \item \textbf{Maintenir} la pression pendant le refroidissement
    \item \textbf{Relâcher} et déplacer au point suivant
\end{enumerate}

\subsection{Espacement des points}

Les points de soudure ne doivent pas être trop proches (effet de shunt) ni trop éloignés (faiblesse mécanique).

\begin{defbox}[Effet de shunt]
    Lorsque deux points de soudure sont trop proches, une partie du courant passe par le premier point au lieu de traverser l'interface à souder. C'est l'effet de \textbf{shunt}. Le nouveau point reçoit moins de courant et est de moins bonne qualité.
\end{defbox}

Espacement recommandé :

\begin{itemize}
    \item \textbf{Minimum} : 10 à 15 fois l'épaisseur de la tôle
    \item \textbf{Optimal} : 20 à 25 fois l'épaisseur
    \item \textbf{Batteries} : 2 points par languette, espacés de 5-8 mm
\end{itemize}

\subsection{Gestion de la chaleur}

Pour les travaux en série ou sur pièces sensibles :

\begin{itemize}
    \item Alterner les zones de soudage (pas de points consécutifs au même endroit)
    \item Laisser refroidir les électrodes régulièrement
    \item Pour les batteries : ne pas enchaîner trop de soudures sur la même cellule
\end{itemize}

\section{Cycle de soudage multi-pulse}

Les soudeuses avancées permettent des cycles multi-impulsions :

\begin{figure}[H]
    \centering
    \begin{tikzpicture}[scale=0.8]
        % Axes
        \draw[->] (0,0) -- (12,0) node[right] {Temps};
        \draw[->] (0,0) -- (0,4) node[above] {Courant};

        % Pré-impulsion
        \fill[info!30] (1,0) rectangle (2,1);
        \draw[info, thick] (1,0) -- (1,1) -- (2,1) -- (2,0);
        \node[below] at (1.5,-0.3) {\small Pré-};
        \node[below] at (1.5,-0.7) {\small impulsion};

        % Pause
        \draw[dashed] (2,0) -- (3,0);

        % Impulsion principale
        \fill[primary!30] (3,0) rectangle (6,3);
        \draw[primary, thick] (3,0) -- (3,3) -- (6,3) -- (6,0);
        \node at (4.5,1.5) {\textbf{Soudage}};

        % Pause
        \draw[dashed] (6,0) -- (7,0);

        % Post-impulsion
        \fill[success!30] (7,0) rectangle (9,2);
        \draw[success, thick] (7,0) -- (7,2) -- (9,2) -- (9,0);
        \node[below] at (8,-0.3) {\small Post-};
        \node[below] at (8,-0.7) {\small traitement};

        % Maintien pression
        \draw[warning, thick, dashed] (0,3.5) -- (11,3.5);
        \node[right] at (11,3.5) {\small Force};
    \end{tikzpicture}
    \caption{Cycle de soudage multi-pulse typique}
\end{figure}

\begin{itemize}
    \item \textbf{Pré-impulsion} : nettoie les surfaces, établit le contact
    \item \textbf{Impulsion principale} : crée le noyau de soudure
    \item \textbf{Post-traitement} : améliore la structure métallurgique
\end{itemize}

\section{Contrôle qualité}

\subsection{Inspection visuelle}

Examinez chaque soudure :

\begin{itemize}
    \item \textbf{Empreinte régulière} : forme circulaire, centrée
    \item \textbf{Pas de projections} : surface propre autour du point
    \item \textbf{Pas de brûlure} : coloration légère acceptable, noir = problème
    \item \textbf{Pas de perforation} : le métal ne doit pas être traversé
\end{itemize}

\begin{figure}[H]
    \centering
    \includegraphics[width=0.75\textwidth]{types-ruptures.png}
    \caption{Différents types de ruptures lors des tests destructifs}
\end{figure}

\subsection{Test destructif (pelage)}

C'est le test de référence pour valider vos paramètres :

\begin{enumerate}
    \item Souder deux échantillons
    \item Tenter de les séparer avec une pince ou en pliant
    \item Observer le mode de rupture :
\end{enumerate}

\begin{center}
\begin{tabular}{L{4cm}L{8cm}}
    \toprule
    \textbf{Mode de rupture} & \textbf{Interprétation} \\
    \midrule
    Arrachement avec bouton & \textcolor{success}{\textbf{Excellent}} - Noyau solide \\
    Déchirement de la tôle & \textcolor{success}{\textbf{Très bon}} - Soudure plus forte que la tôle \\
    Décollage interfacial & \textcolor{warning}{\textbf{Insuffisant}} - Augmenter l'énergie \\
    Pas de tenue & \textcolor{danger}{\textbf{Échec}} - Revoir tous les paramètres \\
    \bottomrule
\end{tabular}
\end{center}

\subsection{Contrôle non destructif}

Pour les applications critiques :

\begin{itemize}
    \item \textbf{Ultrasons} : détecte les défauts internes
    \item \textbf{Courants de Foucault} : vérifie la taille du noyau
    \item \textbf{Thermographie} : contrôle l'homogénéité de chauffe
\end{itemize}

\section{Exercices pratiques}

\begin{exercice}[Première soudure sur acier]
    \textbf{Matériel} : 2 plaques d'acier doux de 1~mm, soudeuse réglable

    \begin{enumerate}
        \item Nettoyer les surfaces à l'alcool
        \item Régler : I = 5~000~A, T = 150~ms
        \item Réaliser une soudure au centre
        \item Tenter de séparer les plaques
        \item Ajuster les paramètres et recommencer
    \end{enumerate}

    \textbf{Objectif} : Obtenir un arrachement avec bouton.
\end{exercice}

\begin{exercice}[Soudure de bande nickel sur cellule 18650]
    \textbf{Matériel} : Cellule 18650 avec languettes, bande Ni 0.15mm, soudeuse batteries

    \begin{enumerate}
        \item Vérifier la tension de la cellule
        \item Positionner la bande de nickel sur la languette positive
        \item Régler : I = 1~000~A, T = 8~ms, double pulse
        \item Souder 2 points espacés de 6~mm
        \item Contrôler visuellement (pas de brûlure)
        \item Tester la tenue mécanique avec précaution
    \end{enumerate}

    \textbf{Objectif} : Soudure propre, bande impossible à arracher à la main.
\end{exercice}

\begin{keypoints}
    \item La préparation des pièces est aussi importante que le soudage lui-même
    \item Commencez toujours par des paramètres conservateurs et augmentez progressivement
    \item Respectez l'espacement minimum entre points (effet de shunt)
    \item Le test de pelage est la meilleure validation de vos paramètres
    \item Un arrachement avec bouton indique une soudure de qualité
\end{keypoints}

\newpage

% ============================================
% CHAPITRE 6 : DIAGNOSTIC ET DÉPANNAGE
% ============================================
\chapter{Diagnostic et Dépannage}

\begin{objectives}
    \item Identifier les défauts de soudure courants
    \item Comprendre leurs causes
    \item Appliquer les actions correctives appropriées
    \item Développer une approche méthodique de résolution
\end{objectives}

Même avec une bonne compréhension de la théorie, vous rencontrerez des problèmes. Ce chapitre vous donne les clés pour les diagnostiquer et les résoudre.

\section{Les 10 défauts les plus courants}

\subsection{1. Soudure froide (collage)}

\textbf{Symptômes} : Les pièces se séparent facilement, pas de bouton visible.

\textbf{Causes possibles} :
\begin{itemize}
    \item Courant insuffisant
    \item Temps trop court
    \item Mauvais contact (surfaces sales, oxydées)
    \item Force excessive (le métal ne fond pas)
\end{itemize}

\textbf{Solutions} :
\begin{itemize}
    \item Augmenter le courant de 10-15\%
    \item Augmenter le temps de soudage
    \item Nettoyer les surfaces
    \item Réduire la force si excessive
\end{itemize}

\subsection{2. Expulsion (projections)}

\textbf{Symptômes} : Projections de métal fondu, éclaboussures autour du point.

\textbf{Causes possibles} :
\begin{itemize}
    \item Courant trop élevé
    \item Force insuffisante
    \item Surfaces contaminées (huile, revêtement)
    \item Électrodes mal alignées
\end{itemize}

\textbf{Solutions} :
\begin{itemize}
    \item Réduire le courant
    \item Augmenter la force de serrage
    \item Nettoyer les surfaces
    \item Réaligner les électrodes
\end{itemize}

\subsection{3. Perforation (brûlure)}

\textbf{Symptômes} : Trou dans la tôle, métal complètement fondu.

\textbf{Causes possibles} :
\begin{itemize}
    \item Courant beaucoup trop élevé
    \item Temps trop long
    \item Électrode usée (contact ponctuel)
    \item Épaisseur de tôle plus faible que prévu
\end{itemize}

\textbf{Solutions} :
\begin{itemize}
    \item Réduire significativement le courant et/ou le temps
    \item Affûter ou remplacer les électrodes
    \item Vérifier l'épaisseur réelle des pièces
\end{itemize}

\subsection{4. Marquage excessif}

\textbf{Symptômes} : Empreinte trop profonde, déformation de la tôle.

\textbf{Causes possibles} :
\begin{itemize}
    \item Force trop élevée
    \item Électrodes trop petites
    \item Température trop élevée
\end{itemize}

\textbf{Solutions} :
\begin{itemize}
    \item Réduire la force
    \item Utiliser des électrodes de diamètre supérieur
    \item Réduire le courant et/ou le temps
\end{itemize}

\subsection{5. Soudure décentrée}

\textbf{Symptômes} : Le point n'est pas au centre des électrodes, forme asymétrique.

\textbf{Causes possibles} :
\begin{itemize}
    \item Électrodes mal alignées
    \item Usure asymétrique des électrodes
    \item Pièces mal positionnées
    \item Rigidité insuffisante du bras
\end{itemize}

\textbf{Solutions} :
\begin{itemize}
    \item Réaligner les électrodes
    \item Affûter ou remplacer les électrodes
    \item Repositionner les pièces
    \item Vérifier le serrage du bras
\end{itemize}

\subsection{6. Noyau trop petit}

\textbf{Symptômes} : Faible résistance mécanique, rupture interfaciale au test.

\textbf{Causes possibles} :
\begin{itemize}
    \item Courant insuffisant
    \item Temps trop court
    \item Diamètre d'électrode trop grand (courant dispersé)
\end{itemize}

\textbf{Solutions} :
\begin{itemize}
    \item Augmenter le courant
    \item Augmenter le temps
    \item Utiliser des électrodes de diamètre adapté
\end{itemize}

\subsection{7. Fissuration à chaud}

\textbf{Symptômes} : Craquelures visibles dans ou autour du noyau.

\textbf{Causes possibles} :
\begin{itemize}
    \item Temps de maintien trop court
    \item Force insuffisante pendant le refroidissement
    \item Matériau sensible (certains aciers haute résistance)
\end{itemize}

\textbf{Solutions} :
\begin{itemize}
    \item Augmenter le temps de maintien sous pression
    \item Maintenir la force pendant le refroidissement
    \item Ajouter une post-impulsion de recuit
\end{itemize}

\subsection{8. Usure rapide des électrodes}

\textbf{Symptômes} : Déformation ou érosion des électrodes après peu de soudures.

\textbf{Causes possibles} :
\begin{itemize}
    \item Courant trop élevé
    \item Refroidissement insuffisant
    \item Matériau d'électrode inadapté
    \item Surfaces des pièces contaminées
\end{itemize}

\textbf{Solutions} :
\begin{itemize}
    \item Réduire le courant
    \item Améliorer le refroidissement
    \item Utiliser des électrodes CuCrZr
    \item Nettoyer les pièces à souder
\end{itemize}

\subsection{9. Collage électrode/pièce}

\textbf{Symptômes} : L'électrode reste collée à la pièce après soudure.

\textbf{Causes possibles} :
\begin{itemize}
    \item Courant trop élevé
    \item Temps trop long
    \item Électrodes sales ou usées
    \item Alliage d'électrode inadapté
\end{itemize}

\textbf{Solutions} :
\begin{itemize}
    \item Réduire le courant et/ou le temps
    \item Nettoyer et affûter les électrodes
    \item Utiliser des électrodes adaptées au matériau
\end{itemize}

\subsection{10. Dispersion de qualité}

\textbf{Symptômes} : Qualité variable d'une soudure à l'autre.

\textbf{Causes possibles} :
\begin{itemize}
    \item Alimentation électrique instable
    \item Usure progressive des électrodes
    \item Variation des pièces (épaisseur, revêtement)
    \item Température ambiante variable
\end{itemize}

\textbf{Solutions} :
\begin{itemize}
    \item Utiliser une alimentation stabilisée
    \item Affûter régulièrement les électrodes
    \item Contrôler les pièces entrantes
    \item Maintenir des conditions de travail stables
\end{itemize}

\section{Méthode de diagnostic}

Face à un problème de soudure, procédez méthodiquement :

\begin{enumerate}
    \item \textbf{Observer} : Décrire précisément le défaut
    \item \textbf{Comparer} : Le défaut est-il nouveau ou récurrent ?
    \item \textbf{Isoler} : Qu'est-ce qui a changé récemment ?
    \item \textbf{Tester} : Modifier un seul paramètre à la fois
    \item \textbf{Documenter} : Noter les résultats pour référence future
\end{enumerate}

\begin{tipbox}[Journal de soudage]
    Tenez un journal de vos soudures avec : date, matériau, épaisseur, paramètres utilisés, résultat. Ce document sera précieux pour reproduire vos réglages et diagnostiquer les problèmes.
\end{tipbox}

\section{Tableau de diagnostic rapide}

\begin{center}
\begin{tabular}{L{3.5cm}C{2cm}C{2cm}C{2cm}C{2cm}}
    \toprule
    \textbf{Problème} & \textbf{Courant} & \textbf{Temps} & \textbf{Force} & \textbf{Autre} \\
    \midrule
    Soudure froide & $\uparrow$ & $\uparrow$ & $\downarrow$ & Nettoyer \\
    Expulsion & $\downarrow$ & -- & $\uparrow$ & Nettoyer \\
    Perforation & $\downarrow\downarrow$ & $\downarrow$ & -- & Électrodes \\
    Marquage excessif & $\downarrow$ & $\downarrow$ & $\downarrow$ & -- \\
    Noyau petit & $\uparrow$ & $\uparrow$ & -- & -- \\
    \bottomrule
\end{tabular}
\end{center}

\textit{Légende : $\uparrow$ = augmenter, $\downarrow$ = diminuer, -- = pas de changement}

\begin{keypoints}
    \item La plupart des défauts sont liés à un déséquilibre courant/temps/force
    \item L'expulsion et la soudure froide sont les défauts les plus courants chez les débutants
    \item Procédez toujours par modifications progressives d'un seul paramètre
    \item L'état des électrodes influence fortement la qualité
    \item Un journal de soudage facilite le diagnostic et la reproductibilité
\end{keypoints}

\newpage

% ============================================
% CHAPITRE 7 : APPLICATIONS PRATIQUES
% ============================================
\chapter{Applications Pratiques}

\begin{objectives}
    \item Découvrir les principales applications du soudage par points
    \item Comprendre les spécificités de chaque domaine
    \item Adapter vos connaissances à votre projet
\end{objectives}

Le soudage par points trouve des applications dans de nombreux domaines. Ce chapitre présente les plus courantes.

\section{Assemblage de batteries lithium}

C'est probablement l'application qui vous a amené à cette formation.

\subsection{Types de configurations}

\begin{figure}[H]
    \centering
    \includegraphics[width=0.8\textwidth]{ustensile-soudure.jpg}
    \caption{Outillage pour l'assemblage de batteries --- Les supports et gabarits facilitent le positionnement précis des cellules.}
\end{figure}

Les packs de batteries se construisent selon différentes configurations :

\begin{itemize}
    \item \textbf{Série (S)} : augmente la tension (ex: 10S = 10 × 3.7V = 37V)
    \item \textbf{Parallèle (P)} : augmente la capacité (ex: 4P = 4 × 2.5Ah = 10Ah)
    \item \textbf{Combinée} : ex: 13S4P = 48V nominal, capacité × 4
\end{itemize}

\subsection{Bonnes pratiques}

\begin{itemize}
    \item Utiliser des cellules de même lot et même capacité
    \item Équilibrer les cellules avant assemblage
    \item Souder sur les languettes, JAMAIS sur le corps
    \item Deux points par connexion minimum
    \item Vérifier la continuité électrique après assemblage
\end{itemize}

\subsection{Dimensionnement des bandes de nickel}

La section du nickel doit être adaptée au courant :

\begin{center}
\begin{tabular}{C{3cm}C{3cm}C{4cm}}
    \toprule
    \textbf{Courant max} & \textbf{Section Ni} & \textbf{Exemple} \\
    \midrule
    10 A & 1.5 mm² & 0.15 × 10 mm \\
    20 A & 3 mm² & 0.15 × 20 mm ou 0.2 × 15 mm \\
    30 A & 4.5 mm² & 0.2 × 23 mm \\
    50 A & 7.5 mm² & 0.25 × 30 mm \\
    \bottomrule
\end{tabular}
\end{center}

\section{Carrosserie automobile}

Le domaine historique du soudage par points.

\subsection{Applications typiques}

\begin{itemize}
    \item Assemblage des panneaux de carrosserie
    \item Réparation après accident
    \item Restauration de véhicules anciens
\end{itemize}

\subsection{Spécificités}

\begin{itemize}
    \item Tôles galvanisées (attention aux fumées de zinc)
    \item Aciers à haute résistance (AHSS) nécessitant des paramètres adaptés
    \item Accès souvent difficile (utilisation de pinces offset)
\end{itemize}

\begin{figure}[H]
    \centering
    \includegraphics[width=0.75\textwidth]{spot-welding-robot2.jpg}
    \caption{Robot de soudage par points industriel --- Ces systèmes réalisent des milliers de points par jour avec une précision constante.}
\end{figure}

\section{Électroménager et mobilier métallique}

\subsection{Applications}

\begin{itemize}
    \item Éviers et plans de travail inox
    \item Casseroles et ustensiles
    \item Armoires métalliques
    \item Rayonnages
\end{itemize}

\subsection{Particularités de l'inox}

L'acier inoxydable a une résistivité élevée mais une faible conductivité thermique :

\begin{itemize}
    \item Courant plus faible que pour l'acier doux
    \item Temps légèrement plus long
    \item Attention au marquage (utiliser des électrodes propres)
\end{itemize}

\section{Électronique et micro-soudage}

\subsection{Applications}

\begin{itemize}
    \item Connexion de fils sur circuits imprimés
    \item Assemblage de capteurs
    \item Bijouterie et horlogerie
    \item Dispositifs médicaux
\end{itemize}

\subsection{Caractéristiques}

\begin{itemize}
    \item Courants très faibles (quelques dizaines à centaines d'ampères)
    \item Temps très courts (1-5 ms)
    \item Électrodes fines (< 2 mm de diamètre)
    \item Souvent sous microscope
\end{itemize}

\section{Aéronautique et spatial}

\subsection{Applications}

\begin{itemize}
    \item Structures légères en aluminium
    \item Réservoirs de carburant
    \item Composants de satellites
\end{itemize}

\subsection{Exigences}

\begin{itemize}
    \item Contrôle qualité 100\% (chaque soudure inspectée)
    \item Traçabilité complète des paramètres
    \item Certifications spécifiques (AS9100, NADCAP)
\end{itemize}

\begin{keypoints}
    \item Le soudage par points s'applique des batteries aux avions
    \item Chaque domaine a ses spécificités (matériaux, normes, contraintes)
    \item L'assemblage de batteries est l'application la plus accessible aux makers
    \item L'industrie automobile reste le plus gros utilisateur en volume
\end{keypoints}

\newpage

% ============================================
% ANNEXES
% ============================================
\appendix

\chapter{Tables de Paramètres}

\section{Acier doux non revêtu}

\begin{center}
\begin{tabular}{C{2cm}C{2.5cm}C{2.5cm}C{2.5cm}C{2.5cm}}
    \toprule
    \textbf{Épaisseur (mm)} & \textbf{Courant (kA)} & \textbf{Temps (ms)} & \textbf{Force (daN)} & \textbf{Ø électrode (mm)} \\
    \midrule
    0.5 & 4 - 5 & 80 - 100 & 100 - 150 & 4 \\
    0.8 & 5 - 7 & 120 - 160 & 150 - 200 & 5 \\
    1.0 & 6 - 8 & 150 - 200 & 200 - 280 & 5 \\
    1.2 & 7 - 9 & 180 - 240 & 250 - 320 & 6 \\
    1.5 & 8 - 10 & 220 - 280 & 300 - 400 & 6 \\
    2.0 & 10 - 12 & 280 - 360 & 400 - 500 & 7 \\
    2.5 & 11 - 14 & 340 - 440 & 500 - 650 & 8 \\
    3.0 & 13 - 16 & 400 - 520 & 600 - 800 & 8 \\
    \bottomrule
\end{tabular}
\end{center}

\section{Acier galvanisé}

Majorer les valeurs de l'acier doux de :
\begin{itemize}
    \item Courant : +10 à 15\%
    \item Temps : +10 à 20\%
    \item Force : identique
\end{itemize}

\section{Acier inoxydable}

\begin{center}
\begin{tabular}{C{2cm}C{2.5cm}C{2.5cm}C{2.5cm}}
    \toprule
    \textbf{Épaisseur (mm)} & \textbf{Courant (kA)} & \textbf{Temps (ms)} & \textbf{Force (daN)} \\
    \midrule
    0.5 & 3.5 - 4.5 & 100 - 140 & 80 - 120 \\
    1.0 & 5 - 7 & 180 - 260 & 150 - 220 \\
    1.5 & 7 - 9 & 260 - 360 & 220 - 320 \\
    2.0 & 8 - 11 & 340 - 480 & 300 - 420 \\
    \bottomrule
\end{tabular}
\end{center}

\section{Nickel pour batteries}

\begin{center}
\begin{tabular}{C{2cm}C{2.5cm}C{2.5cm}C{3cm}}
    \toprule
    \textbf{Épaisseur (mm)} & \textbf{Courant (A)} & \textbf{Temps (ms)} & \textbf{Remarques} \\
    \midrule
    0.10 & 600 - 900 & 3 - 6 & Bande fine \\
    0.15 & 800 - 1~300 & 5 - 12 & Standard \\
    0.20 & 1~000 - 1~600 & 8 - 18 & Haute capacité \\
    0.25 & 1~200 - 2~000 & 10 - 25 & Fort courant \\
    0.30 & 1~500 - 2~500 & 15 - 30 & Très fort courant \\
    \bottomrule
\end{tabular}
\end{center}

\chapter{Glossaire}

\begin{description}
    \item[Courant de soudage] Intensité électrique traversant l'assemblage pendant la soudure (en kA).

    \item[Cycle de travail] Rapport entre le temps de soudage actif et le temps total. Un cycle de 50\% signifie 50\% du temps en soudage, 50\% en refroidissement.

    \item[Effet Joule] Phénomène de production de chaleur lors du passage d'un courant électrique dans un conducteur.

    \item[Électrode] Pièce en cuivre (ou alliage) qui transmet le courant et la pression aux pièces à souder.

    \item[Expulsion] Projection de métal fondu hors de la zone de soudure (défaut).

    \item[Force de serrage] Pression exercée par les électrodes sur les pièces (en daN ou kN).

    \item[Noyau (nugget)] Zone fondue puis solidifiée formant la liaison entre les pièces.

    \item[Résistance dynamique] Résistance électrique de l'assemblage, qui varie pendant le cycle de soudure.

    \item[RSW] Resistance Spot Welding --- Soudage par résistance par points.

    \item[Shunt] Passage du courant par un chemin autre que celui souhaité (ex: point de soudure voisin).

    \item[ZAT] Zone Affectée Thermiquement --- Région autour du noyau où le métal a subi un cycle thermique sans fondre.
\end{description}

\chapter{Ressources Complémentaires}

\section{Normes et standards}

\begin{itemize}
    \item \textbf{ISO 4063} : Nomenclature des procédés de soudage
    \item \textbf{ISO 14373} : Soudage par résistance --- Procédure de soudage par points
    \item \textbf{ISO 18278} : Soudabilité --- Évaluation de la soudabilité
    \item \textbf{AWS D8.1} : Spécification pour le soudage automobile
    \item \textbf{AWS D8.9} : Recommended practices for test methods for evaluating the resistance spot welding behavior
\end{itemize}

\section{Sites web de référence}

\begin{itemize}
    \item \textbf{www.spotweldingpro.com} : Notre site avec tutoriels et mises à jour
    \item \textbf{www.rwma.org} : Resistance Welding Manufacturing Alliance
    \item \textbf{www.aws.org} : American Welding Society
    \item \textbf{www.twi-global.com} : The Welding Institute (UK)
\end{itemize}

\section{Formations complémentaires}

\begin{itemize}
    \item \textbf{Niveau 2} : Maîtrise Avancée --- Optimisation, batteries lithium, multi-pulse
    \item \textbf{Niveau 3} : Expert --- Métallurgie, automatisation, certification
\end{itemize}

\vspace{2cm}

\begin{center}
    \textcolor{primary}{\rule{0.5\textwidth}{1pt}}

    \vspace{1cm}

    {\Large\textbf{Merci d'avoir suivi cette formation !}}

    \vspace{0.5cm}

    N'hésitez pas à nous contacter pour toute question.

    \vspace{1cm}

    \textbf{www.spotweldingpro.com}

    \textcolor{primary}{\rule{0.5\textwidth}{1pt}}
\end{center}

\end{document}

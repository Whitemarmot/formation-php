% Préambule commun pour toutes les formations
% Spot Welding Pro - Formations PDF Premium
% Version 2.0 - Style Professionnel Clair

\documentclass[11pt,a4paper,oneside]{book}

% ============================================
% ENCODAGE ET LANGUE
% ============================================
\usepackage[utf8]{inputenc}
\usepackage[T1]{fontenc}
\usepackage[french]{babel}

% ============================================
% GÉOMÉTRIE DE PAGE
% ============================================
\usepackage[
    top=2.5cm,
    bottom=2.5cm,
    left=2.5cm,
    right=2.5cm,
    headheight=24pt
]{geometry}

% ============================================
% POLICES - Source Sans Pro (moderne et lisible)
% ============================================
\usepackage{fontspec}
\setmainfont{Source Sans Pro}[
    UprightFont = *-Regular,
    BoldFont = *-Bold,
    ItalicFont = *-It,
    BoldItalicFont = *-BoldIt
]
\setsansfont{Source Sans Pro}[
    UprightFont = *-Regular,
    BoldFont = *-Bold,
    ItalicFont = *-It,
    BoldItalicFont = *-BoldIt
]
\setmonofont{Source Code Pro}[
    Scale=0.9
]

% ============================================
% COULEURS - Palette Professionnelle Claire
% ============================================
\usepackage{xcolor}

% Couleurs principales
\definecolor{primary}{HTML}{DC2626}      % Rouge professionnel
\definecolor{secondary}{HTML}{1E40AF}    % Bleu foncé
\definecolor{accent}{HTML}{F59E0B}       % Orange/ambre pour accents

% Couleurs de fond
\definecolor{background}{HTML}{FFFFFF}   % Blanc pur
\definecolor{lightgray}{HTML}{F8FAFC}    % Gris très clair pour boîtes
\definecolor{cardgray}{HTML}{F1F5F9}     % Gris clair pour cartes

% Couleurs de texte
\definecolor{textdark}{HTML}{1F2937}     % Gris très foncé (quasi noir)
\definecolor{textmuted}{HTML}{6B7280}    % Gris moyen pour texte secondaire
\definecolor{textlight}{HTML}{9CA3AF}    % Gris clair

% Couleurs sémantiques
\definecolor{success}{HTML}{059669}      % Vert succès
\definecolor{warning}{HTML}{D97706}      % Orange avertissement
\definecolor{danger}{HTML}{DC2626}       % Rouge danger
\definecolor{info}{HTML}{0284C7}         % Bleu info

% ============================================
% GRAPHIQUES ET IMAGES
% ============================================
\usepackage{graphicx}
\usepackage{float}
\usepackage{wrapfig}
\usepackage{caption}
\usepackage{subcaption}

% Chemin des images
\graphicspath{{../commun/images/}{./images/}}

% Style des captions
\captionsetup{
    font={small},
    labelfont={bf,color=primary},
    textfont={color=textmuted},
    margin=1cm
}

% ============================================
% TABLEAUX
% ============================================
\usepackage{booktabs}
\usepackage{longtable}
\usepackage{multirow}
\usepackage{makecell}
\usepackage{colortbl}
\usepackage{array}

% Style des tableaux
\renewcommand{\arraystretch}{1.3}
\newcolumntype{L}[1]{>{\raggedright\arraybackslash}p{#1}}
\newcolumntype{C}[1]{>{\centering\arraybackslash}p{#1}}
\newcolumntype{R}[1]{>{\raggedleft\arraybackslash}p{#1}}

% ============================================
% LISTES
% ============================================
\usepackage{enumitem}
\setlist[itemize]{
    leftmargin=*,
    itemsep=0.5em,
    labelsep=0.8em
}
\setlist[enumerate]{
    leftmargin=*,
    itemsep=0.5em,
    labelsep=0.8em
}
% Puce personnalisée
\setlist[itemize,1]{label={\textcolor{primary}{\textbullet}}}
\setlist[itemize,2]{label={\textcolor{secondary}{\textendash}}}

% ============================================
% MATHÉMATIQUES ET UNITÉS
% ============================================
\usepackage{amsmath}
\usepackage{amssymb}
\usepackage{siunitx}
\sisetup{
    locale=FR,
    output-decimal-marker={,},
    group-separator={\,},
    per-mode=symbol
}

% Unités personnalisées
\DeclareSIUnit{\ampere}{A}
\DeclareSIUnit{\kiloampere}{kA}
\DeclareSIUnit{\milliseconde}{ms}
\DeclareSIUnit{\newton}{N}
\DeclareSIUnit{\kilonewton}{kN}
\DeclareSIUnit{\ohm}{\Omega}
\DeclareSIUnit{\milliohm}{m\Omega}
\DeclareSIUnit{\microohm}{\micro\Omega}

% ============================================
% CODE ET ALGORITHMES
% ============================================
\usepackage{listings}
\lstset{
    basicstyle=\ttfamily\small\color{textdark},
    keywordstyle=\color{primary}\bfseries,
    commentstyle=\color{textmuted}\itshape,
    stringstyle=\color{success},
    numbers=left,
    numberstyle=\tiny\color{textlight},
    numbersep=10pt,
    frame=single,
    frameround=tttt,
    backgroundcolor=\color{lightgray},
    rulecolor=\color{cardgray},
    breaklines=true,
    showstringspaces=false,
    tabsize=4
}

% ============================================
% BOÎTES COLORÉES (tcolorbox)
% ============================================
\usepackage{tcolorbox}
\tcbuselibrary{skins,breakable,hooks}

% Boîte d'information avec icône
\newtcolorbox{infobox}[1][À noter]{
    enhanced,
    colback=info!5!white,
    colframe=info,
    coltitle=white,
    fonttitle=\bfseries,
    left=12pt,
    right=12pt,
    top=10pt,
    bottom=10pt,
    arc=6pt,
    boxrule=1.5pt,
    title={\raisebox{-0.1em}{\begin{tikzpicture}[baseline=-0.5ex]\fill[white] (0,0) circle (0.4em);\node[info, font=\bfseries\tiny] at (0,0) {i};\end{tikzpicture}}~#1},
    breakable,
    shadow={2pt}{-2pt}{0pt}{black!20}
}

% Boîte d'avertissement avec icône
\newtcolorbox{warningbox}[1][Attention]{
    enhanced,
    colback=warning!8!white,
    colframe=warning,
    coltitle=white,
    fonttitle=\bfseries,
    left=12pt,
    right=12pt,
    top=10pt,
    bottom=10pt,
    arc=6pt,
    boxrule=1.5pt,
    title={\raisebox{-0.1em}{\begin{tikzpicture}[baseline=-0.5ex]\fill[white] (0,0) -- (0.5em,0.8em) -- (-0.5em,0.8em) -- cycle;\node[warning, font=\bfseries\tiny] at (0,0.4em) {!};\end{tikzpicture}}~#1},
    breakable,
    shadow={2pt}{-2pt}{0pt}{black!20}
}

% Boîte de danger avec icône
\newtcolorbox{dangerbox}[1][Danger]{
    enhanced,
    colback=danger!8!white,
    colframe=danger,
    coltitle=white,
    fonttitle=\bfseries,
    left=12pt,
    right=12pt,
    top=10pt,
    bottom=10pt,
    arc=6pt,
    boxrule=1.5pt,
    title={\raisebox{-0.1em}{\begin{tikzpicture}[baseline=-0.5ex]\fill[white] (0,0) -- (0.4em,0.2em) -- (0.4em,0.6em) -- (0,0.8em) -- (-0.4em,0.6em) -- (-0.4em,0.2em) -- cycle;\node[danger, font=\bfseries\tiny] at (0,0.4em) {X};\end{tikzpicture}}~#1},
    breakable,
    shadow={2pt}{-2pt}{0pt}{black!20}
}

% Boîte de conseil/astuce avec icône ampoule
\newtcolorbox{tipbox}[1][Conseil]{
    enhanced,
    colback=success!8!white,
    colframe=success,
    coltitle=white,
    fonttitle=\bfseries,
    left=12pt,
    right=12pt,
    top=10pt,
    bottom=10pt,
    arc=6pt,
    boxrule=1.5pt,
    title={\raisebox{-0.1em}{\begin{tikzpicture}[baseline=-0.5ex]\fill[white] (0,0.2em) circle (0.4em);\fill[white] (-0.15em,-0.1em) rectangle (0.15em,0.1em);\draw[success, line width=0.8pt] (0,0.2em) circle (0.25em);\draw[success, line width=0.6pt] (-0.1em,0.35em) -- (0.1em,0.35em);\draw[success, line width=0.6pt] (0,0.2em) -- (0,0.5em);\end{tikzpicture}}~#1},
    breakable,
    shadow={2pt}{-2pt}{0pt}{black!20}
}

% Boîte de définition avec icône livre
\newtcolorbox{defbox}[1][Définition]{
    enhanced,
    colback=secondary!5!white,
    colframe=secondary,
    coltitle=white,
    fonttitle=\bfseries,
    left=12pt,
    right=12pt,
    top=10pt,
    bottom=10pt,
    arc=6pt,
    boxrule=1.5pt,
    title={\raisebox{-0.1em}{\begin{tikzpicture}[baseline=-0.5ex]\fill[white] (-0.4em,0) rectangle (0.4em,0.7em);\draw[secondary, line width=0.8pt] (0,0) -- (0,0.7em);\draw[secondary, line width=0.8pt] (-0.4em,0) -- (-0.4em,0.7em) -- (0.4em,0.7em) -- (0.4em,0);\end{tikzpicture}}~#1},
    breakable,
    shadow={2pt}{-2pt}{0pt}{black!20}
}

% Boîte pour formules importantes
\newtcolorbox{formulabox}{
    enhanced,
    colback=lightgray,
    colframe=primary,
    boxrule=2pt,
    arc=4pt,
    left=15pt,
    right=15pt,
    top=12pt,
    bottom=12pt,
    shadow={2pt}{-2pt}{0pt}{black!15}
}

% Nouvelle boîte : Résumé avec icône
\newtcolorbox{resumebox}[1][En résumé]{
    enhanced,
    colback=primary!5!white,
    colframe=primary,
    coltitle=white,
    fonttitle=\bfseries,
    left=12pt,
    right=12pt,
    top=10pt,
    bottom=10pt,
    arc=6pt,
    boxrule=1.5pt,
    title={\raisebox{-0.1em}{\begin{tikzpicture}[baseline=-0.5ex]\fill[white] (0,0) circle (0.4em);\draw[primary, line width=0.8pt] (-0.2em,0) -- (0.2em,0);\draw[primary, line width=0.8pt] (-0.15em,0.15em) -- (0.15em,0.15em);\draw[primary, line width=0.8pt] (-0.2em,-0.15em) -- (0.2em,-0.15em);\end{tikzpicture}}~#1},
    breakable,
    shadow={2pt}{-2pt}{0pt}{black!20}
}

% Nouvelle boîte : Exemple pratique
\newtcolorbox{examplebox}[1][Exemple pratique]{
    enhanced,
    colback=accent!5!white,
    colframe=accent,
    coltitle=white,
    fonttitle=\bfseries,
    left=12pt,
    right=12pt,
    top=10pt,
    bottom=10pt,
    arc=6pt,
    boxrule=1.5pt,
    title={\raisebox{-0.1em}{\begin{tikzpicture}[baseline=-0.5ex]\draw[white, line width=1.2pt] (-0.3em,0) -- (0.3em,0) -- (0.15em,0.4em) -- cycle;\end{tikzpicture}}~#1},
    breakable,
    shadow={2pt}{-2pt}{0pt}{black!20}
}

% Nouvelle boîte : Paramètres techniques
\newtcolorbox{parambox}[1][Paramètres]{
    enhanced,
    colback=secondary!3!white,
    colframe=secondary!70!black,
    coltitle=white,
    fonttitle=\bfseries,
    left=12pt,
    right=12pt,
    top=10pt,
    bottom=10pt,
    arc=6pt,
    boxrule=1.5pt,
    title={\raisebox{-0.1em}{\begin{tikzpicture}[baseline=-0.5ex, scale=0.9]\fill[white] (0,0) circle (0.3em);\foreach \a in {0,60,...,300} {\fill[white, rotate=\a] (0,0.3em) rectangle (0.1em,0.45em);}\fill[secondary!70!black] (0,0) circle (0.12em);\end{tikzpicture}}~#1},
    breakable,
    shadow={2pt}{-2pt}{0pt}{black!20}
}

% ============================================
% EN-TÊTES ET PIEDS DE PAGE
% ============================================
\usepackage{fancyhdr}
\pagestyle{fancy}
\fancyhf{}
\fancyhead[L]{\small\textcolor{textmuted}{\leftmark}}
\fancyhead[R]{\small\textcolor{primary}{\textbf{Spot Welding Pro}}}
\fancyfoot[C]{\small\textcolor{textmuted}{\thepage}}
\renewcommand{\headrulewidth}{0.5pt}
\renewcommand{\headrule}{\hbox to\headwidth{\color{primary}\leaders\hrule height \headrulewidth\hfill}}
\renewcommand{\footrulewidth}{0pt}

% Style pour pages de chapitre
\fancypagestyle{plain}{
    \fancyhf{}
    \fancyfoot[C]{\small\textcolor{textmuted}{\thepage}}
    \renewcommand{\headrulewidth}{0pt}
}

% ============================================
% TITRES DE CHAPITRES ET SECTIONS
% ============================================
\usepackage{titlesec}

% Chapitres
\titleformat{\chapter}[display]
    {\normalfont\huge\bfseries\color{primary}}
    {\chaptertitlename\ \thechapter}
    {20pt}
    {\Huge}
\titlespacing*{\chapter}{0pt}{-20pt}{40pt}

% Sections
\titleformat{\section}
    {\normalfont\Large\bfseries\color{secondary}}
    {\thesection}
    {1em}
    {}
\titlespacing*{\section}{0pt}{3.5ex plus 1ex minus .2ex}{2.3ex plus .2ex}

% Sous-sections
\titleformat{\subsection}
    {\normalfont\large\bfseries\color{textdark}}
    {\thesubsection}
    {1em}
    {}

% Sous-sous-sections
\titleformat{\subsubsection}
    {\normalfont\normalsize\bfseries\color{textdark}}
    {\thesubsubsection}
    {1em}
    {}

% ============================================
% ESPACEMENT
% ============================================
\usepackage{setspace}
\onehalfspacing
\setlength{\parskip}{0.5em}
\setlength{\parindent}{0pt}

% ============================================
% TABLE DES MATIÈRES
% ============================================
\usepackage{tocloft}
\renewcommand{\cftchapfont}{\bfseries\color{primary}}
\renewcommand{\cftsecfont}{\color{textdark}}
\renewcommand{\cftsubsecfont}{\color{textmuted}}
\renewcommand{\cftchapleader}{\cftdotfill{\cftdotsep}}
\setlength{\cftbeforechapskip}{1em}

% ============================================
% LIENS HYPERTEXTE
% ============================================
\usepackage[
    colorlinks=true,
    linkcolor=primary,
    urlcolor=secondary,
    citecolor=success,
    bookmarks=true,
    bookmarksnumbered=true,
    pdfstartview=FitH
]{hyperref}

% ============================================
% RÉFÉRENCES ET BIBLIOGRAPHIE
% ============================================
\usepackage{cleveref}
\crefname{figure}{figure}{figures}
\crefname{table}{tableau}{tableaux}
\crefname{chapter}{chapitre}{chapitres}
\crefname{section}{section}{sections}

\usepackage{footnote}
\usepackage[style=numeric,sorting=none]{biblatex}

% ============================================
% GLOSSAIRE ET INDEX
% ============================================
\usepackage[acronym,toc]{glossaries}
\makeglossaries

\usepackage{makeidx}
\makeindex

% ============================================
% DIAGRAMMES TikZ
% ============================================
\usepackage{tikz}
\usetikzlibrary{
    shapes,
    shapes.geometric,
    shapes.symbols,
    arrows,
    arrows.meta,
    positioning,
    calc,
    decorations.pathreplacing,
    patterns,
    shadows,
    backgrounds,
    fit
}

% Styles TikZ personnalisés
\tikzset{
    block/.style={
        rectangle,
        draw=secondary,
        fill=lightgray,
        text=textdark,
        minimum width=3cm,
        minimum height=1cm,
        align=center,
        rounded corners=4pt
    },
    arrow/.style={
        ->,
        >=stealth,
        thick,
        color=primary
    },
    electrode/.style={
        fill=accent!70,
        draw=accent!80!black
    },
    metal/.style={
        fill=gray!40,
        draw=gray!60
    },
    weld/.style={
        fill=primary!60,
        draw=primary
    }
}

% ============================================
% ICÔNES ET PICTOGRAMMES (TikZ)
% ============================================

% Icône d'information (i dans un cercle)
\newcommand{\iconinfo}{%
    \begin{tikzpicture}[baseline=-0.5ex]
        \fill[info] (0,0) circle (0.4em);
        \node[white, font=\bfseries\tiny] at (0,0) {i};
    \end{tikzpicture}%
}

% Icône d'avertissement (triangle avec !)
\newcommand{\iconwarning}{%
    \begin{tikzpicture}[baseline=-0.5ex]
        \fill[warning] (0,0) -- (0.4em,0.7em) -- (-0.4em,0.7em) -- cycle;
        \node[white, font=\bfseries\tiny] at (0,0.35em) {!};
    \end{tikzpicture}%
}

% Icône de danger (hexagone avec X)
\newcommand{\icondanger}{%
    \begin{tikzpicture}[baseline=-0.5ex]
        \fill[danger] (0,0) -- (0.35em,0.2em) -- (0.35em,0.5em) -- (0,0.7em) -- (-0.35em,0.5em) -- (-0.35em,0.2em) -- cycle;
        \node[white, font=\bfseries\tiny] at (0,0.35em) {X};
    \end{tikzpicture}%
}

% Icône de conseil (ampoule)
\newcommand{\icontip}{%
    \begin{tikzpicture}[baseline=-0.5ex]
        \fill[success] (0,0.15em) circle (0.35em);
        \fill[success] (-0.15em,0) rectangle (0.15em,-0.2em);
        \draw[white, line width=0.5pt] (0,0.15em) -- (0,0.4em);
        \draw[white, line width=0.5pt] (-0.2em,0.35em) -- (0.2em,0.35em);
    \end{tikzpicture}%
}

% Icône de définition (livre)
\newcommand{\icondef}{%
    \begin{tikzpicture}[baseline=-0.5ex]
        \fill[secondary] (-0.35em,0) rectangle (0.35em,0.6em);
        \draw[white, line width=0.5pt] (0,0) -- (0,0.6em);
    \end{tikzpicture}%
}

% Icône de validation (check)
\newcommand{\iconcheck}{%
    \begin{tikzpicture}[baseline=-0.5ex]
        \fill[success] (0,0) circle (0.4em);
        \draw[white, line width=1pt] (-0.2em,0) -- (-0.05em,-0.15em) -- (0.2em,0.2em);
    \end{tikzpicture}%
}

% Icône d'erreur (croix)
\newcommand{\iconerror}{%
    \begin{tikzpicture}[baseline=-0.5ex]
        \fill[danger] (0,0) circle (0.4em);
        \draw[white, line width=1pt] (-0.15em,-0.15em) -- (0.15em,0.15em);
        \draw[white, line width=1pt] (-0.15em,0.15em) -- (0.15em,-0.15em);
    \end{tikzpicture}%
}

% Icône d'éclair/énergie
\newcommand{\iconbolt}{%
    \begin{tikzpicture}[baseline=-0.5ex]
        \fill[accent] (0.1em,0.6em) -- (-0.2em,0.25em) -- (0.05em,0.25em) -- (-0.1em,0) -- (0.2em,0.35em) -- (-0.05em,0.35em) -- cycle;
    \end{tikzpicture}%
}

% Icône d'outil/clé
\newcommand{\icontool}{%
    \begin{tikzpicture}[baseline=-0.5ex]
        \fill[textmuted] (0,0) circle (0.2em);
        \fill[textmuted] (0.15em,-0.15em) rectangle (0.5em,0.05em);
    \end{tikzpicture}%
}

% Icône de paramètre/engrenage
\newcommand{\icongear}{%
    \begin{tikzpicture}[baseline=-0.5ex, scale=0.8]
        \fill[secondary] (0,0) circle (0.25em);
        \foreach \a in {0,45,...,315} {
            \fill[secondary, rotate=\a] (0,0.25em) rectangle (0.1em,0.4em);
        }
        \fill[white] (0,0) circle (0.1em);
    \end{tikzpicture}%
}

% Icône de température/thermomètre
\newcommand{\icontemp}{%
    \begin{tikzpicture}[baseline=-0.5ex]
        \fill[danger] (0,0) circle (0.2em);
        \fill[danger] (-0.08em,0.1em) rectangle (0.08em,0.5em);
        \fill[white] (-0.04em,0.2em) rectangle (0.04em,0.45em);
    \end{tikzpicture}%
}

% Icône de chronomètre/temps
\newcommand{\icontime}{%
    \begin{tikzpicture}[baseline=-0.5ex]
        \draw[secondary, line width=1pt] (0,0.2em) circle (0.35em);
        \draw[secondary, line width=0.8pt] (0,0.2em) -- (0,0.4em);
        \draw[secondary, line width=0.8pt] (0,0.2em) -- (0.15em,0.2em);
    \end{tikzpicture}%
}

% Icône de force/pression
\newcommand{\iconforce}{%
    \begin{tikzpicture}[baseline=-0.5ex]
        \draw[primary, line width=1.5pt, ->, >=stealth] (0,0.5em) -- (0,0);
        \draw[primary, line width=1.5pt, ->, >=stealth] (0,-0.1em) -- (0,0.4em);
    \end{tikzpicture}%
}

% ============================================
% CIRCUITS ÉLECTRIQUES
% ============================================
\usepackage{circuitikz}

% ============================================
% GRAPHIQUES DE DONNÉES
% ============================================
\usepackage{pgfplots}
\pgfplotsset{
    compat=1.18,
    every axis/.append style={
        xlabel style={color=textdark},
        ylabel style={color=textdark},
        tick label style={color=textmuted},
        legend style={
            fill=lightgray,
            draw=cardgray
        },
        grid=major,
        grid style={color=cardgray}
    }
}

% ============================================
% COMMANDES PERSONNALISÉES
% ============================================
\newcommand{\formation}[1]{\textbf{\textcolor{primary}{#1}}}
\newcommand{\parametre}[1]{\texttt{\textcolor{secondary}{#1}}}
\newcommand{\valeur}[2]{\SI{#1}{#2}}
\newcommand{\marque}[1]{\textit{#1}}
\newcommand{\attention}[1]{\textcolor{warning}{\textbf{#1}}}
\newcommand{\danger}[1]{\textcolor{danger}{\textbf{#1}}}
\newcommand{\important}[1]{\textcolor{primary}{\textbf{#1}}}

% Loi de Joule
\newcommand{\joule}{Q = R \cdot I^2 \cdot t}

% ============================================
% ENVIRONNEMENTS PERSONNALISÉS
% ============================================

% Exercices avec icône crayon
\newcounter{exercice}[chapter]
\newenvironment{exercice}[1][]{%
    \refstepcounter{exercice}%
    \begin{tcolorbox}[
        enhanced,
        colback=accent!5!white,
        colframe=accent,
        coltitle=white,
        fonttitle=\bfseries,
        title={\raisebox{-0.1em}{\begin{tikzpicture}[baseline=-0.5ex]\fill[white] (0,0) -- (0.5em,0) -- (0.6em,0.6em) -- (0.1em,0.6em) -- cycle;\fill[white] (0.55em,0.55em) -- (0.65em,0.65em) -- (0.55em,0.75em) -- (0.45em,0.65em) -- cycle;\end{tikzpicture}}~Exercice \theexercice\ifx&#1&\else~: #1\fi},
        breakable,
        arc=6pt,
        boxrule=1.5pt,
        shadow={2pt}{-2pt}{0pt}{black!20}
    ]
}{%
    \end{tcolorbox}
}

% Études de cas avec icône loupe
\newenvironment{casestudy}[1][]{%
    \begin{tcolorbox}[
        enhanced,
        colback=primary!5!white,
        colframe=primary,
        coltitle=white,
        fonttitle=\bfseries,
        title={\raisebox{-0.1em}{\begin{tikzpicture}[baseline=-0.5ex]\draw[white, line width=1pt] (0,0.3em) circle (0.3em);\draw[white, line width=1pt] (0.2em,0.1em) -- (0.5em,-0.2em);\end{tikzpicture}}~Étude de cas\ifx&#1&\else~: #1\fi},
        breakable,
        arc=6pt,
        boxrule=1.5pt,
        shadow={2pt}{-2pt}{0pt}{black!20}
    ]
}{%
    \end{tcolorbox}
}

% Points clés avec icône clé
\newenvironment{keypoints}{%
    \begin{tcolorbox}[
        enhanced,
        colback=secondary!5!white,
        colframe=secondary,
        coltitle=white,
        fonttitle=\bfseries,
        title={\raisebox{-0.1em}{\begin{tikzpicture}[baseline=-0.5ex]\fill[white] (0,0.4em) circle (0.25em);\fill[white] (-0.08em,0) rectangle (0.08em,0.2em);\fill[white] (-0.15em,-0.05em) rectangle (0.15em,0.05em);\fill[white] (-0.15em,-0.15em) rectangle (0.15em,-0.1em);\end{tikzpicture}}~Points clés à retenir},
        breakable,
        arc=6pt,
        boxrule=1.5pt,
        shadow={2pt}{-2pt}{0pt}{black!20}
    ]
    \begin{itemize}[leftmargin=*]
}{%
    \end{itemize}
    \end{tcolorbox}
}

% Objectifs de chapitre avec icône cible
\newenvironment{objectives}{%
    \begin{tcolorbox}[
        enhanced,
        colback=success!5!white,
        colframe=success,
        coltitle=white,
        fonttitle=\bfseries,
        title={\raisebox{-0.1em}{\begin{tikzpicture}[baseline=-0.5ex]\draw[white, line width=1pt] (0,0.3em) circle (0.35em);\draw[white, line width=0.8pt] (0,0.3em) circle (0.2em);\fill[white] (0,0.3em) circle (0.08em);\end{tikzpicture}}~Objectifs du chapitre},
        arc=6pt,
        boxrule=1.5pt,
        shadow={2pt}{-2pt}{0pt}{black!20}
    ]
    \begin{itemize}[leftmargin=*]
}{%
    \end{itemize}
    \end{tcolorbox}
}

% Nouveau : Checklist avec icônes de validation
\newenvironment{checklist}{%
    \begin{tcolorbox}[
        enhanced,
        colback=success!3!white,
        colframe=success!70!black,
        coltitle=white,
        fonttitle=\bfseries,
        title={\raisebox{-0.1em}{\begin{tikzpicture}[baseline=-0.5ex]\fill[white] (-0.35em,0) rectangle (0.35em,0.6em);\draw[success!70!black, line width=0.8pt] (-0.15em,0.2em) -- (-0.05em,0.1em) -- (0.15em,0.4em);\end{tikzpicture}}~Checklist},
        arc=6pt,
        boxrule=1.5pt,
        shadow={2pt}{-2pt}{0pt}{black!20}
    ]
    \begin{itemize}[leftmargin=*, label={\textcolor{success}{\checkmark}}]
}{%
    \end{itemize}
    \end{tcolorbox}
}

% Nouveau : Tableau récapitulatif
\newenvironment{recapbox}[1][Récapitulatif]{%
    \begin{tcolorbox}[
        enhanced,
        colback=cardgray,
        colframe=textdark,
        coltitle=white,
        fonttitle=\bfseries,
        title={\raisebox{-0.1em}{\begin{tikzpicture}[baseline=-0.5ex]\fill[white] (-0.35em,0) rectangle (0.35em,0.5em);\draw[white, line width=0.5pt] (-0.35em,0.25em) -- (0.35em,0.25em);\draw[white, line width=0.5pt] (0,0) -- (0,0.5em);\end{tikzpicture}}~#1},
        arc=6pt,
        boxrule=1.5pt,
        shadow={2pt}{-2pt}{0pt}{black!20}
    ]
}{%
    \end{tcolorbox}
}

% Nouveau : Note de bas de page mise en avant
\newenvironment{sidenote}{%
    \begin{tcolorbox}[
        enhanced,
        colback=lightgray,
        colframe=textmuted,
        boxrule=0pt,
        leftrule=3pt,
        arc=0pt,
        left=10pt,
        right=10pt,
        top=8pt,
        bottom=8pt
    ]
    \small\textcolor{textmuted}
}{%
    \end{tcolorbox}
}

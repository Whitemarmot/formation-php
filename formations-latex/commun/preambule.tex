% Préambule commun pour toutes les formations
% Spot Welding Pro - Formations PDF Premium

\documentclass[11pt,a4paper,oneside]{book}

% Encodage et langue
\usepackage[utf8]{inputenc}
\usepackage[T1]{fontenc}
\usepackage[french]{babel}

% Géométrie de page
\usepackage[
    top=2.5cm,
    bottom=2.5cm,
    left=2.5cm,
    right=2.5cm,
    headheight=14pt
]{geometry}

% Polices (utilise les polices TeX Live par défaut)
\usepackage{fontspec}
\setmainfont{Latin Modern Roman}
\setsansfont{Latin Modern Sans}
\setmonofont{Latin Modern Mono}

% Couleurs
\usepackage{xcolor}
\definecolor{primary}{HTML}{E94560}
\definecolor{secondary}{HTML}{F39C12}
\definecolor{darkbg}{HTML}{0F0F1A}
\definecolor{darkcard}{HTML}{16213E}
\definecolor{textcolor}{HTML}{EAEAEA}
\definecolor{mutedtext}{HTML}{9CA3AF}
\definecolor{success}{HTML}{22C55E}
\definecolor{warning}{HTML}{F59E0B}
\definecolor{danger}{HTML}{EF4444}

% Graphiques et images
\usepackage{graphicx}
\usepackage{float}
\usepackage{wrapfig}
\usepackage{caption}
\usepackage{subcaption}

% Tableaux
\usepackage{booktabs}
\usepackage{longtable}
\usepackage{multirow}
\usepackage{makecell}
\usepackage{colortbl}
\usepackage{array}

% Listes
\usepackage{enumitem}
\setlist[itemize]{leftmargin=*,itemsep=0.5em}
\setlist[enumerate]{leftmargin=*,itemsep=0.5em}

% Mathématiques
\usepackage{amsmath}
\usepackage{amssymb}
\usepackage{siunitx}
\sisetup{
    locale=FR,
    output-decimal-marker={,},
    group-separator={\,}
}

% Code et algorithmes
\usepackage{listings}
\lstset{
    basicstyle=\ttfamily\small,
    keywordstyle=\color{primary}\bfseries,
    commentstyle=\color{mutedtext}\itshape,
    stringstyle=\color{success},
    numbers=left,
    numberstyle=\tiny\color{mutedtext},
    numbersep=10pt,
    frame=single,
    frameround=tttt,
    backgroundcolor=\color{darkcard},
    rulecolor=\color{mutedtext},
    breaklines=true,
    showstringspaces=false
}

% Boîtes colorées
\usepackage{tcolorbox}
\tcbuselibrary{skins,breakable}

% Boîte d'information
\newtcolorbox{infobox}[1][]{
    enhanced,
    colback=darkcard,
    colframe=primary,
    coltitle=white,
    fonttitle=\bfseries,
    left=10pt,
    right=10pt,
    top=10pt,
    bottom=10pt,
    arc=4pt,
    boxrule=1pt,
    title=#1
}

% Boîte d'avertissement
\newtcolorbox{warningbox}[1][Attention]{
    enhanced,
    colback=warning!10!darkbg,
    colframe=warning,
    coltitle=white,
    fonttitle=\bfseries,
    left=10pt,
    right=10pt,
    top=10pt,
    bottom=10pt,
    arc=4pt,
    boxrule=1pt,
    title=#1
}

% Boîte de danger
\newtcolorbox{dangerbox}[1][Danger]{
    enhanced,
    colback=danger!10!darkbg,
    colframe=danger,
    coltitle=white,
    fonttitle=\bfseries,
    left=10pt,
    right=10pt,
    top=10pt,
    bottom=10pt,
    arc=4pt,
    boxrule=1pt,
    title=#1
}

% Boîte de conseil
\newtcolorbox{tipbox}[1][Conseil]{
    enhanced,
    colback=success!10!darkbg,
    colframe=success,
    coltitle=white,
    fonttitle=\bfseries,
    left=10pt,
    right=10pt,
    top=10pt,
    bottom=10pt,
    arc=4pt,
    boxrule=1pt,
    title=#1
}

% En-têtes et pieds de page
\usepackage{fancyhdr}
\pagestyle{fancy}
\fancyhf{}
\fancyhead[L]{\small\textcolor{mutedtext}{\leftmark}}
\fancyhead[R]{\small\textcolor{mutedtext}{Spot Welding Pro}}
\fancyfoot[C]{\small\textcolor{mutedtext}{\thepage}}
\renewcommand{\headrulewidth}{0.5pt}
\renewcommand{\headrule}{\hbox to\headwidth{\color{primary}\leaders\hrule height \headrulewidth\hfill}}
\renewcommand{\footrulewidth}{0pt}

% Titres de chapitres
\usepackage{titlesec}
\titleformat{\chapter}[display]
    {\normalfont\huge\bfseries\color{primary}}
    {\chaptertitlename\ \thechapter}
    {20pt}
    {\Huge}
\titleformat{\section}
    {\normalfont\Large\bfseries\color{primary}}
    {\thesection}
    {1em}
    {}
\titleformat{\subsection}
    {\normalfont\large\bfseries}
    {\thesubsection}
    {1em}
    {}
\titleformat{\subsubsection}
    {\normalfont\normalsize\bfseries}
    {\thesubsubsection}
    {1em}
    {}

% Espacement
\usepackage{setspace}
\onehalfspacing

% Table des matières
\usepackage{tocloft}
\renewcommand{\cftchapfont}{\bfseries\color{primary}}
\renewcommand{\cftsecfont}{\color{textcolor}}
\renewcommand{\cftsubsecfont}{\color{mutedtext}}
\renewcommand{\cftchapleader}{\cftdotfill{\cftdotsep}}

% Liens hypertexte
\usepackage[
    colorlinks=true,
    linkcolor=primary,
    urlcolor=secondary,
    citecolor=success,
    bookmarks=true,
    bookmarksnumbered=true
]{hyperref}

% Références croisées améliorées
\usepackage{cleveref}

% Notes de bas de page
\usepackage{footnote}

% Bibliographie
\usepackage[style=numeric,sorting=none]{biblatex}

% Glossaire
\usepackage[acronym,toc]{glossaries}
\makeglossaries

% Index
\usepackage{makeidx}
\makeindex

% Diagrammes TikZ
\usepackage{tikz}
\usetikzlibrary{
    shapes,
    arrows,
    positioning,
    calc,
    decorations.pathreplacing,
    patterns
}

% Circuits électriques
\usepackage{circuitikz}

% Graphiques de données
\usepackage{pgfplots}
\pgfplotsset{compat=1.18}

% Commandes personnalisées
\newcommand{\formation}[1]{\textbf{\textcolor{primary}{#1}}}
\newcommand{\parametre}[1]{\texttt{#1}}
\newcommand{\valeur}[2]{\SI{#1}{#2}}
\newcommand{\marque}[1]{\textit{#1}}
\newcommand{\attention}[1]{\textcolor{warning}{\textbf{#1}}}
\newcommand{\danger}[1]{\textcolor{danger}{\textbf{#1}}}

% Unités personnalisées
\DeclareSIUnit{\ampere}{A}
\DeclareSIUnit{\kiloampere}{kA}
\DeclareSIUnit{\milliseconde}{ms}
\DeclareSIUnit{\newton}{N}
\DeclareSIUnit{\kilonewton}{kN}
\DeclareSIUnit{\ohm}{\Omega}
\DeclareSIUnit{\milliohm}{m\Omega}

% Commande pour la loi de Joule
\newcommand{\joule}{Q = R \cdot I^2 \cdot t}

% Environnement pour les exercices
\newcounter{exercice}[chapter]
\newenvironment{exercice}[1][]{%
    \refstepcounter{exercice}%
    \begin{tcolorbox}[
        enhanced,
        colback=darkcard,
        colframe=secondary,
        coltitle=white,
        fonttitle=\bfseries,
        title={Exercice \theexercice\ifx&#1&\else: #1\fi},
        breakable
    ]
}{%
    \end{tcolorbox}
}

% Environnement pour les études de cas
\newenvironment{casestudy}[1][]{%
    \begin{tcolorbox}[
        enhanced,
        colback=darkcard,
        colframe=primary,
        coltitle=white,
        fonttitle=\bfseries,
        title={Étude de cas\ifx&#1&\else: #1\fi},
        breakable
    ]
}{%
    \end{tcolorbox}
}
